\documentclass[11pt, landscape]{article}   	% use "amsart" instead of "article" for AMSLaTeX format
\usepackage{geometry}                		% See geometry.pdf to learn the layout options. There are lots.
\geometry{letterpaper}                   		% ... or a4paper or a5paper or ... 
%\geometry{landscape}                		% Activate for for rotated page geometry
%\usepackage[parfill]{parskip}    		% Activate to begin paragraphs with an empty line rather than an indent
\usepackage{graphicx}				% Use pdf, png, jpg, or eps§ with pdflatex; use eps in DVI mode
								% TeX will automatically convert eps --> pdf in pdflatex		
\usepackage{amssymb}

\title{Brief information of the genes}
\author{AD project}
%\date{}							% Activate to display a given date or no date

\begin{document}
\maketitle
%\section{}
%\subsection{}


% latex table generated in R 3.1.1 by xtable 1.7-4 package
% Sat Jan 24 17:21:56 2015
\begin{table}[ht]
\centering
\tiny
\begin{tabular}{rlp{3cm}lp{12cm}}
  \hline
 & query & name & hg & summary \\ 
  \hline
1 & 1700047I17Rik2 & RIKEN cDNA 1700047I17 gene 2 &  & Mus:NA Hg:NA \\ 
  2 & 2310036O22Rik & RIKEN cDNA 2310036O22 gene & C19orf43 & Mus:NA Hg:NA \\ 
  3 & 2900079G21Rik & RIKEN cDNA 2900079G21 gene &  & Mus:NA Hg:NA \\ 
  5 & 9430020K01Rik & RIKEN cDNA 9430020K01 gene & KIAA1462 & Mus:NA Hg:NA \\ 
  6 & A030009H04Rik &  &  & Mus:NA Hg:NA \\ 
  7 & AC182748.1 &  &  & Mus:NA Hg:NA \\ 
  8 & Actn1 & actinin, alpha 1 & ACTN1 & Mus:NA Hg:Alpha actinins belong to the spectrin gene superfamily which represents a diverse group of cytoskeletal proteins, including the alpha and beta spectrins and dystrophins. Alpha actinin is an actin-binding protein with multiple roles in different cell types. In nonmuscle cells, the cytoskeletal isoform is found along microfilament bundles and adherens-type junctions, where it is involved in binding actin to the membrane. In contrast, skeletal, cardiac, and smooth muscle isoforms are localized to the Z-disc and analogous dense bodies, where they help anchor the myofibrillar actin filaments. This gene encodes a nonmuscle, cytoskeletal, alpha actinin isoform and maps to the same site as the structurally similar erythroid beta spectrin gene. Three transcript variants encoding different isoforms have been found for this gene. \\ 
  9 & Akap5 & A kinase (PRKA) anchor protein 5 & AKAP5 & Mus:NA Hg:The A-kinase anchor proteins (AKAPs) are a group of structurally diverse proteins, which have the common function of binding to the regulatory subunit of protein kinase A (PKA) and confining the holoenzyme to discrete locations within the cell. This gene encodes a member of the AKAP family. The encoded protein binds to the RII-beta regulatory subunit of PKA, and also to protein kinase C and the phosphatase calcineurin. It is predominantly expressed in cerebral cortex and may anchor the PKA protein at postsynaptic densities (PSD) and be involved in the regulation of postsynaptic events. It is also expressed in T lymphocytes and may function to inhibit interleukin-2 transcription by disrupting calcineurin-dependent dephosphorylation of NFAT. \\ 
  10 & Aldh9a1 & aldehyde dehydrogenase 9, subfamily A1 & ALDH9A1 & Mus:NA Hg:This protein belongs to the aldehyde dehydrogenase family of proteins. It has a high activity for oxidation of gamma-aminobutyraldehyde and other amino aldehydes. The enzyme catalyzes the dehydrogenation of gamma-aminobutyraldehyde to gamma-aminobutyric acid (GABA). This isozyme is a tetramer of identical 54-kD subunits. \\ 
  11 & Ank3 & ankyrin 3, epithelial & ANK3 & Mus:This gene encodes a member of the ankyrin protein family. Ankyrins link integral membrane proteins to the spectrin-based cytoskeleton. Ankyrin family members share a protein structure which includes three independently folded domains: the N-terminal ankyrin repeat domain, the central spectrin-binding domain, and the C-terminal rod domain. This ankyrin functions as the major ankyrin in the kidney and may play a role in the polarized distribution of many integral membrane proteins to specific subcellular sites. Alternative splicing of this gene results in multiple transcript variants encoding different isoforms. Hg:Ankyrins are a family of proteins that are believed to link the integral membrane proteins to the underlying spectrin-actin cytoskeleton and play key roles in activities such as cell motility, activation, proliferation, contact, and the maintenance of specialized membrane domains. Multiple isoforms of ankyrin with different affinities for various target proteins are expressed in a tissue-specific, developmentally regulated manner. Most ankyrins are typically composed of three structural domains: an amino-terminal domain containing multiple ankyrin repeats; a central region with a highly conserved spectrin binding domain; and a carboxy-terminal regulatory domain which is the least conserved and subject to variation. Ankyrin 3 is an immunologically distinct gene product from ankyrins 1 and 2, and was originally found at the axonal initial segment and nodes of Ranvier of neurons in the central and peripheral nervous systems. Multiple transcript variants encoding different isoforms have been found for this gene. \\ 
  12 & Ankrd17 & ankyrin repeat domain 17 & ANKRD17 & Mus:This gene encodes a protein with ankyrin repeats, which are associated with protein-protein interactions. Studies suggest that this protein is involved in liver development. Two transcript variants encoding different isoforms have been found for this gene. Hg:This gene encodes a protein with ankyrin repeats, which are associated with protein-protein interactions. Studies in mice suggest that this protein is involved in liver development. Three transcript variants encoding different isoforms have been found for this gene. \\ 
  13 & Ankrd52 & ankyrin repeat domain 52 & ANKRD52 & Mus:NA Hg:NA \\ 
  14 & Ap2b1 & adaptor-related protein complex 2, beta 1 subunit &  & Mus:NA Hg:NA \\ 

   \hline
\end{tabular}
\end{table}

\begin{table}[ht]
\centering
\tiny
\begin{tabular}{rlp{3cm}lp{12cm}}
  \hline
 & query & name & hg & summary \\ 
  \hline


  15 & Apod & apolipoprotein D & APOD & Mus:The protein encoded by this gene is a component of high-density lipoprotein (HDL), but is unique in that it shares greater structural similarity to lipocalin than to other members of the apolipoprotein family, and has a wider tissue expression pattern. The encoded protein is involved in lipid metabolism, and ablation of this gene results in defects in triglyceride metabolism. Elevated levels of this gene product have been observed in multiple tissues of Niemann-Pick disease mouse models, as well as in some tumors. Alternative splicing results in multiple transcript variants. Hg:This gene encodes a component of high density lipoprotein that has no marked similarity to other apolipoprotein sequences. It has a high degree of homology to plasma retinol-binding protein and other members of the alpha 2 microglobulin protein superfamily of carrier proteins, also known as lipocalins. This glycoprotein is closely associated with the enzyme lecithin:cholesterol acyltransferase - an enzyme involved in lipoprotein metabolism. \\ 
  16 & Apoe & apolipoprotein E & APOE & Mus:NA Hg:The protein encoded by this gene is a major apoprotein of the chylomicron. It binds to a specific liver and peripheral cell receptor, and is essential for the normal catabolism of triglyceride-rich lipoprotein constituents. This gene maps to chromosome 19 in a cluster with the related apolipoprotein C1 and C2 genes. Mutations in this gene result in familial dysbetalipoproteinemia, or type III hyperlipoproteinemia (HLP III), in which increased plasma cholesterol and triglycerides are the consequence of impaired clearance of chylomicron and VLDL remnants. Alternative splicing results in multiple transcript variants. \\ 
  18 & Arhgef7 & Rho guanine nucleotide exchange factor (GEF7) & ARHGEF7 & Mus:NA Hg:Rho GTPases play a fundamental role in numerous cellular processes triggered by extracellular stimuli that work through G protein coupled receptors. The encoded protein belongs to a family of cytoplasmic proteins that activate the Ras-like family of Rho proteins by exchanging bound GDP for GTP. It forms a complex with the small GTP binding protein Rac1 and recruits Rac1 to membrane ruffles and to focal adhesions. This protein can induce membrane ruffling. Multiple alternatively spliced transcript variants encoding different isoforms have been described for this gene. \\ 
  19 & Arid1a & AT rich interactive domain 1A (SWI-like) & ARID1A & Mus:NA Hg:This gene encodes a member of the SWI/SNF family, whose members have helicase and ATPase activities and are thought to regulate transcription of certain genes by altering the chromatin structure around those genes. The encoded protein is part of the large ATP-dependent chromatin remodeling complex SNF/SWI, which is required for transcriptional activation of genes normally repressed by chromatin. It possesses at least two conserved domains that could be important for its function. First, it has a DNA-binding domain that can specifically bind an AT-rich DNA sequence known to be recognized by a SNF/SWI complex at the beta-globin locus. Second, the C-terminus of the protein can stimulate glucocorticoid receptor-dependent transcriptional activation. It is thought that the protein encoded by this gene confers specificity to the SNF/SWI complex and may recruit the complex to its targets through either protein-DNA or protein-protein interactions. Two transcript variants encoding different isoforms have been found for this gene. \\ 
  20 & Asap1 & ArfGAP with SH3 domain, ankyrin repeat and PH domain1 & ASAP1 & Mus:NA Hg:This gene encodes an ADP-ribosylation factor (ARF) GTPase-activating protein. The GTPase-activating activity is stimulated by phosphatidylinositol 4,5-biphosphate (PIP2), and is greater towards ARF1 and ARF5, and lesser for ARF6. This gene maybe involved in regulation of membrane trafficking and cytoskeleton remodeling. Alternatively spliced transcript variants encoding different isoforms have been found for this gene. \\ 
  21 & Asl & argininosuccinate lyase & ASL & Mus:NA Hg:This gene encodes a member of the lyase 1 family. The encoded protein forms a cytosolic homotetramer and primarily catalyzes the reversible hydrolytic cleavage of argininosuccinate into arginine and fumarate, an essential step in the liver in detoxifying ammonia via the urea cycle. Mutations in this gene result in the autosomal recessive disorder argininosuccinic aciduria, or argininosuccinic acid lyase deficiency. A nontranscribed pseudogene is also located on the long arm of chromosome 22. Alternatively spliced transcript variants encoding different isoforms have been described. \\ 
  23 & Ass1 & argininosuccinate synthetase 1 & ASS1 & Mus:NA Hg:The protein encoded by this gene catalyzes the penultimate step of the arginine biosynthetic pathway. There are approximately 10 to 14 copies of this gene including the pseudogenes scattered across the human genome, among which the one located on chromosome 9 appears to be the only functional gene for argininosuccinate synthetase. Mutations in the chromosome 9 copy of this gene cause citrullinemia. Two transcript variants encoding the same protein have been found for this gene. \\ 
  24 & Atp2a2 & ATPase, Ca++ transporting, cardiac muscle, slow twitch 2 & ATP2A2 & Mus:NA Hg:This gene encodes one of the SERCA Ca(2+)-ATPases, which are intracellular pumps located in the sarcoplasmic or endoplasmic reticula of muscle cells. This enzyme catalyzes the hydrolysis of ATP coupled with the translocation of calcium from the cytosol into the sarcoplasmic reticulum lumen, and is involved in regulation of the contraction/relaxation cycle. Mutations in this gene cause Darier-White disease, also known as keratosis follicularis, an autosomal dominant skin disorder characterized by loss of adhesion between epidermal cells and abnormal keratinization. Alternative splicing results in multiple transcript variants encoding different isoforms. \\ 
 
   \hline
\end{tabular}
\end{table}

\begin{table}[ht]
\centering
\tiny
\begin{tabular}{rlp{3cm}lp{12cm}}
  \hline
 & query & name & hg & summary \\ 
  \hline


  25 & Atp2b2 & ATPase, Ca++ transporting, plasma membrane 2 & ATP2B2 & Mus:NA Hg:The protein encoded by this gene belongs to the family of P-type primary ion transport ATPases characterized by the formation of an aspartyl phosphate intermediate during the reaction cycle. These enzymes remove bivalent calcium ions from eukaryotic cells against very large concentration gradients and play a critical role in intracellular calcium homeostasis. The mammalian plasma membrane calcium ATPase isoforms are encoded by at least four separate genes and the diversity of these enzymes is further increased by alternative splicing of transcripts. The expression of different isoforms and splice variants is regulated in a developmental, tissue- and cell type-specific manner, suggesting that these pumps are functionally adapted to the physiological needs of particular cells and tissues. This gene encodes the plasma membrane calcium ATPase isoform 2. Alternatively spliced transcript variants encoding different isoforms have been identified. \\ 
  27 & C1qa & complement component 1, q subcomponent, alpha polypeptide & C1QA & Mus:NA Hg:This gene encodes a major constituent of the human complement subcomponent C1q. C1q associates with C1r and C1s in order to yield the first component of the serum complement system. Deficiency of C1q has been associated with lupus erythematosus and glomerulonephritis. C1q is composed of 18 polypeptide chains: six A-chains, six B-chains, and six C-chains. Each chain contains a collagen-like region located near the N terminus and a C-terminal globular region. The A-, B-, and C-chains are arranged in the order A-C-B on chromosome 1. This gene encodes the A-chain polypeptide of human complement subcomponent C1q. \\ 
  28 & C1qc & complement component 1, q subcomponent, C chain & C1QC & Mus:NA Hg:This gene encodes a major constituent of the human complement subcomponent C1q. C1q associates with C1r and C1s in order to yield the first component of the serum complement system. A deficiency in C1q has been associated with lupus erythematosus and glomerulonephritis. C1q is composed of 18 polypeptide chains: six A-chains, six B-chains, and six C-chains. Each chain contains a collagen-like region located near the N-terminus, and a C-terminal globular region. The A-, B-, and C-chains are arranged in the order A-C-B on chromosome 1. This gene encodes the C-chain polypeptide of human complement subcomponent C1q. Alternatively spliced transcript variants that encode the same protein have been found for this gene. \\ 
  29 & C1qtnf4 & C1q and tumor necrosis factor related protein 4 & C1QTNF4 & Mus:NA Hg:NA \\ 
  30 & C4b & complement component 4B (Chido blood group) & C4A & Mus:NA Hg:This gene encodes the acidic form of complement factor 4, part of the classical activation pathway. The protein is expressed as a single chain precursor which is proteolytically cleaved into a trimer of alpha, beta, and gamma chains prior to secretion. The trimer provides a surface for interaction between the antigen-antibody complex and other complement components. The alpha chain is cleaved to release C4 anaphylatoxin, an antimicrobial peptide and a mediator of local inflammation. Deficiency of this protein is associated with systemic lupus erythematosus and type I diabetes mellitus. This gene localizes to the major histocompatibility complex (MHC) class III region on chromosome 6. Varying haplotypes of this gene cluster exist, such that individuals may have 1, 2, or 3 copies of this gene. Two transcript variants encoding different isoforms have been found for this gene. \\ 
  31 & Cacna1e & calcium channel, voltage-dependent, R type, alpha 1E subunit & CACNA1E & Mus:This gene encodes an integral membrane protein that belongs to the calcium channel alpha-1 subunits family. Voltage-sensitive calcium channels mediate the entry of calcium ions into excitable cells and are also involved in a variety of calcium-dependent processes. Voltage-dependent calcium channels are multi-subunit complexes, comprised of alpha-1, alpha-2, beta and delta subunits in a 1:1:1:1 ratio. The isoform alpha-1E gives rise to R-type calcium currents and belongs to the high-voltage activated group. Calcium channels containing the alpha-1E subunit may be involved in the modulation of neuronal firing patterns, an important component of information processing. Hg:Voltage-dependent calcium channels are multisubunit complexes consisting of alpha-1, alpha-2, beta, and delta subunits in a 1:1:1:1 ratio. These channels mediate the entry of calcium ions into excitable cells, and are also involved in a variety of calcium-dependent processes, including muscle contraction, hormone or neurotransmitter release, gene expression, cell motility, cell division and cell death. This gene encodes the alpha-1E subunit of the R-type calcium channels, which belong to the 'high-voltage activated' group that maybe involved in the modulation of firing patterns of neurons important for information processing. Alternatively spliced transcript variants encoding different isoforms have been described for this gene. \\ 
  32 & Camk4 & calcium/calmodulin-dependent protein kinase IV & CAMK4 & Mus:NA Hg:The product of this gene belongs to the serine/threonine protein kinase family, and to the Ca(2+)/calmodulin-dependent protein kinase subfamily. This enzyme is a multifunctional serine/threonine protein kinase with limited tissue distribution, that has been implicated in transcriptional regulation in lymphocytes, neurons and male germ cells. \\ 
  33 & Camsap1 & calmodulin regulated spectrin-associated protein 1 & CAMSAP1 & Mus:NA Hg:NA \\ 

   \hline
\end{tabular}
\end{table}

\begin{table}[ht]
\centering
\tiny
\begin{tabular}{rlp{3cm}lp{12cm}}
  \hline
 & query & name & hg & summary \\ 
  \hline

  34 & Car4 & carbonic anhydrase 4 & CA4 & Mus:NA Hg:Carbonic anhydrases (CAs) are a large family of zinc metalloenzymes that catalyze the reversible hydration of carbon dioxide. They participate in a variety of biological processes, including respiration, calcification, acid-base balance, bone resorption, and the formation of aqueous humor, cerebrospinal fluid, saliva, and gastric acid. They show extensive diversity in tissue distribution and in their subcellular localization. This gene encodes a glycosylphosphatidyl-inositol-anchored membrane isozyme expressed on the luminal surfaces of pulmonary (and certain other) capillaries and proximal renal tubules. Its exact function is not known; however, it may have a role in inherited renal abnormalities of bicarbonate transport. \\ 
  35 & Cbx7 & chromobox 7 & CBX7 & Mus:NA Hg:NA \\ 
  36 & Ccdc124 & coiled-coil domain containing 124 & CCDC124 & Mus:NA Hg:NA \\ 
  37 & Ccdc85b & coiled-coil domain containing 85B & CCDC85B & Mus:NA Hg:Hepatitis delta virus (HDV) is a pathogenic human virus whose RNA genome and replication cycle resemble those of plant viroids. Delta-interacting protein A (DIPA), a cellular gene product, has been found to have homology to hepatitis delta virus antigen (HDAg). DIPA interacts with the viral antigen, HDAg, and can affect HDV replication in vitro. \\ 
  38 & Ccnd2 & cyclin D2 & CCND2 & Mus:NA Hg:The protein encoded by this gene belongs to the highly conserved cyclin family, whose members are characterized by a dramatic periodicity in protein abundance through the cell cycle. Cyclins function as regulators of CDK kinases. Different cyclins exhibit distinct expression and degradation patterns which contribute to the temporal coordination of each mitotic event. This cyclin forms a complex with CDK4 or CDK6 and functions as a regulatory subunit of the complex, whose activity is required for cell cycle G1/S transition. This protein has been shown to interact with and be involved in the phosphorylation of tumor suppressor protein Rb. Knockout studies of the homologous gene in mouse suggest the essential roles of this gene in ovarian granulosa and germ cell proliferation. High level expression of this gene was observed in ovarian and testicular tumors. Mutations in this gene are associated with megalencephaly-polymicrogyria-polydactyly-hydrocephalus syndrome 3 (MPPH3). \\ 
  39 & Cdkn2aipnl & CDKN2A interacting protein N-terminal like & CDKN2AIPNL & Mus:NA Hg:NA \\ 
  40 & Chchd10 & coiled-coil-helix-coiled-coil-helix domain containing 10 & CHCHD10 & Mus:NA Hg:This gene encodes a mitochondrial protein that is enriched at cristae junctions in the intermembrane space. It may play a role in cristae morphology maintenance or oxidative phosphorylation. Mutations in this gene cause frontotemporal dementia and/or amyotrophic lateral sclerosis-2. Alternative splicing of this gene results in multiple transcript variants. Related pseudogenes have been identified on chromosomes 7 and 19. \\ 
  41 & Cit & citron & CIT & Mus:NA Hg:This gene encodes a serine/threonine-protein kinase that functions in cell division. Together with the kinesin KIF14, this protein localizes to the central spindle and midbody, and functions to promote efficient cytokinesis. This protein is involved in central nervous system development. Polymorphisms in this gene are associated with bipolar disorder and risk for schizophrenia. Alternative splicing results in multiple transcript variants. \\ 
  42 & Cldn11 & claudin 11 & CLDN11 & Mus:This gene encodes a member of the claudin family. Claudins are integral membrane proteins and components of tight junction strands. Tight junction strands serve as a physical barrier to prevent solutes and water from passing freely through the paracellular space between epithelial or endothelial cell sheets, and also play critical roles in maintaining cell polarity and signal transductions. The protein encoded by this gene is a major component of CNS (central nervous system) myelin and plays an important role in regulating proliferation and migration of oligodendrocytes. The basal cell tight junctions in stria vascularis are primarily composed of this protein, and the gene-null mice suffer severe deafness. This protein is also an obligatory protein for tight junction formation and barrier integrity in the testis and the gene deficiency results in loss of the Sertoli cell epithelial phenotype in the testis. Hg:This gene encodes a member of the claudin family. Claudins are integral membrane proteins and components of tight junction strands. Tight junction strands serve as a physical barrier to prevent solutes and water from passing freely through the paracellular space between epithelial or endothelial cell sheets, and also play critical roles in maintaining cell polarity and signal transductions. The protein encoded by this gene is a major component of central nervous system (CNS) myelin and plays an important role in regulating proliferation and migration of oligodendrocytes. Mouse studies showed that the gene deficiency results in deafness and loss of the Sertoli cell epithelial phenotype in the testis. This protein is a tight junction protein at the human blood-testis barrier (BTB), and the BTB disruption is related to a dysfunction of this gene. Alternatively spliced transcript variants encoding different isoforms have been identified. \\ 

   \hline
\end{tabular}
\end{table}

\begin{table}[ht]
\centering
\tiny
\begin{tabular}{rlp{3cm}lp{12cm}}
  \hline
 & query & name & hg & summary \\ 
  \hline

  43 & Col4a1 & collagen, type IV, alpha 1 & COL4A1 & Mus:NA Hg:This gene encodes a type IV collagen alpha protein. Type IV collagen proteins are integral components of basement membranes. This gene shares a bidirectional promoter with a paralogous gene on the opposite strand. The protein consists of an amino-terminal 7S domain, a triple-helix forming collagenous domain, and a carboxy-terminal non-collagenous domain. It functions as part of a heterotrimer and interacts with other extracellular matrix components such as perlecans, proteoglycans, and laminins. In addition, proteolytic cleavage of the non-collagenous carboxy-terminal domain results in a biologically active fragment known as arresten, which has anti-angiogenic and tumor suppressor properties. Mutations in this gene cause porencephaly, cerebrovascular disease, and renal and muscular defects. Alternative splicing results in multiple transcript variants. \\ 
  44 & Coprs & coordinator of PRMT5, differentiation stimulator & COPRS & Mus:NA Hg:NA \\ 
  45 & Cox17 & cytochrome c oxidase assembly protein 17 & COX17 & Mus:NA Hg:Cytochrome c oxidase (COX), the terminal component of the mitochondrial respiratory chain, catalyzes the electron transfer from reduced cytochrome c to oxygen. This component is a heteromeric complex consisting of 3 catalytic subunits encoded by mitochondrial genes and multiple structural subunits encoded by nuclear genes. The mitochondrially-encoded subunits function in electron transfer, and the nuclear-encoded subunits may function in the regulation and assembly of the complex. This nuclear gene encodes a protein which is not a structural subunit, but may be involved in the recruitment of copper to mitochondria for incorporation into the COX apoenzyme. This protein shares 92\% amino acid sequence identity with mouse and rat Cox17 proteins. This gene is no longer considered to be a candidate gene for COX deficiency. A pseudogene COX17P has been found on chromosome 13. \\ 
  46 & Crebbp & CREB binding protein & CREBBP & Mus:NA Hg:This gene is ubiquitously expressed and is involved in the transcriptional coactivation of many different transcription factors. First isolated as a nuclear protein that binds to cAMP-response element binding protein (CREB), this gene is now known to play critical roles in embryonic development, growth control, and homeostasis by coupling chromatin remodeling to transcription factor recognition. The protein encoded by this gene has intrinsic histone acetyltransferase activity and also acts as a scaffold to stabilize additional protein interactions with the transcription complex. This protein acetylates both histone and non-histone proteins. This protein shares regions of very high sequence similarity with protein p300 in its bromodomain, cysteine-histidine-rich regions, and histone acetyltransferase domain. Mutations in this gene cause Rubinstein-Taybi syndrome (RTS). Chromosomal translocations involving this gene have been associated with acute myeloid leukemia. Alternative splicing results in multiple transcript variants encoding different isoforms. \\ 
  47 & Cst3 & cystatin C & CST3 & Mus:NA Hg:The cystatin superfamily encompasses proteins that contain multiple cystatin-like sequences. Some of the members are active cysteine protease inhibitors, while others have lost or perhaps never acquired this inhibitory activity. There are three inhibitory families in the superfamily, including the type 1 cystatins (stefins), type 2 cystatins and the kininogens. The type 2 cystatin proteins are a class of cysteine proteinase inhibitors found in a variety of human fluids and secretions, where they appear to provide protective functions. The cystatin locus on chromosome 20 contains the majority of the type 2 cystatin genes and pseudogenes. This gene is located in the cystatin locus and encodes the most abundant extracellular inhibitor of cysteine proteases, which is found in high concentrations in biological fluids and is expressed in virtually all organs of the body. A mutation in this gene has been associated with amyloid angiopathy. Expression of this protein in vascular wall smooth muscle cells is severely reduced in both atherosclerotic and aneurysmal aortic lesions, establishing its role in vascular disease. In addition, this protein has been shown to have an antimicrobial function, inhibiting the replication of herpes simplex virus. Alternative splicing results in multiple transcript variants encoding a single protein. \\ 
  48 & Ctss & cathepsin S & CTSS & Mus:NA Hg:The protein encoded by this gene, a member of the peptidase C1 family, is a lysosomal cysteine proteinase that may participate in the degradation of antigenic proteins to peptides for presentation on MHC class II molecules. The encoded protein can function as an elastase over a broad pH range in alveolar macrophages. Alternatively spliced transcript variants encoding distinct isoforms have been found for this gene. \\ 
  49 & Ctsz & cathepsin Z & CTSZ & Mus:NA Hg:The protein encoded by this gene is a lysosomal cysteine proteinase and member of the peptidase C1 family. It exhibits both carboxy-monopeptidase and carboxy-dipeptidase activities. The encoded protein has also been known as cathepsin X and cathepsin P. This gene is expressed ubiquitously in cancer cell lines and primary tumors and, like other members of this family, may be involved in tumorigenesis. \\ 
  50 & Cyb5r1 & cytochrome b5 reductase 1 & CYB5R1 & Mus:NA Hg:NA \\ 
  51 & D430041D05Rik & RIKEN cDNA D430041D05 gene & KIAA1549L & Mus:NA Hg:NA \\ 

   \hline
\end{tabular}
\end{table}

\begin{table}[ht]
\centering
\tiny
\begin{tabular}{rlp{3cm}lp{12cm}}
  \hline
 & query & name & hg & summary \\ 
  \hline

  52 & Dio2 & deiodinase, iodothyronine, type II & DIO2 & Mus:The protein encoded by this gene belongs to the iodothyronine deiodinase family. It activates thyroid hormone by converting the prohormone thyroxine (T4) by outer ring deiodination (ORD) to bioactive 3,3',5-triiodothyronine (T3). Knockout studies in mice suggest that this gene may play a role in brown adipose tissue lipogenesis, auditory function, and bone formation. This protein contains selenocysteine (Sec) residues encoded by the UGA codon, which normally signals translation termination. The 3' UTR of Sec-containing genes have a common stem-loop structure, the sec insertion sequence (SECIS), which is necessary for the recognition of UGA as a Sec codon rather than as a stop signal. Hg:The protein encoded by this gene belongs to the iodothyronine deiodinase family. It activates thyroid hormone by converting the prohormone thyroxine (T4) by outer ring deiodination (ORD) to bioactive 3,3',5-triiodothyronine (T3). It is highly expressed in the thyroid, and may contribute significantly to the relative increase in thyroidal T3 production in patients with Graves disease and thyroid adenomas. This protein contains selenocysteine (Sec) residues encoded by the UGA codon, which normally signals translation termination. The 3' UTR of Sec-containing genes have a common stem-loop structure, the sec insertion sequence (SECIS), which is necessary for the recognition of UGA as a Sec codon rather than as a stop signal. Alternative splicing results in multiple transcript variants encoding different isoforms. \\ 
  53 & Dnajb11 & DnaJ (Hsp40) homolog, subfamily B, member 11 & DNAJB11 & Mus:NA Hg:This gene encodes a soluble glycoprotein of the endoplasmic reticulum (ER) lumen that functions as a co-chaperone of binding immunoglobulin protein, a 70 kilodalton heat shock protein chaperone required for the proper folding and assembly of proteins in the ER. The encoded protein contains a highly conserved J domain of about 70 amino acids with a characteristic His-Pro-Asp (HPD) motif and may regulate the activity of binding immunoglobulin protein by stimulating ATPase activity. \\ 
  55 & Dnmt3a & DNA methyltransferase 3A & DNMT3A & Mus:NA Hg:CpG methylation is an epigenetic modification that is important for embryonic development, imprinting, and X-chromosome inactivation. Studies in mice have demonstrated that DNA methylation is required for mammalian development. This gene encodes a DNA methyltransferase that is thought to function in de novo methylation, rather than maintenance methylation. The protein localizes to the cytoplasm and nucleus and its expression is developmentally regulated. Alternative splicing results in multiple transcript variants encoding different isoforms. \\ 
  57 & Dpcd & deleted in primary ciliary dyskinesia & DPCD & Mus:NA Hg:NA \\ 
  58 & Dpysl3 & dihydropyrimidinase-like 3 & DPYSL3 & Mus:This gene encodes a protein that belongs to the TUC (TOAD-64/Ulip/CRMP) family of proteins. Members of this family are phosphoproteins that function in axonal guidance and neuronal differentiation during development and regeneration of the nervous system. A mutation in the human gene is associated with amyotrophic lateral sclerosis. Alternative splicing results in multiple transcript variants encoding different isoforms. Hg:NA \\ 
  59 & Ech1 & enoyl coenzyme A hydratase 1, peroxisomal & ECH1 & Mus:NA Hg:This gene encodes a member of the hydratase/isomerase superfamily. The gene product shows high sequence similarity to enoyl-coenzyme A (CoA) hydratases of several species, particularly within a conserved domain characteristic of these proteins. The encoded protein, which contains a C-terminal peroxisomal targeting sequence, localizes to the peroxisome. The rat ortholog, which localizes to the matrix of both the peroxisome and mitochondria, can isomerize 3-trans,5-cis-dienoyl-CoA to 2-trans,4-trans-dienoyl-CoA, indicating that it is a delta3,5-delta2,4-dienoyl-CoA isomerase. This enzyme functions in the auxiliary step of the fatty acid beta-oxidation pathway. Expression of the rat gene is induced by peroxisome proliferators. \\ 
  60 & Erc1 & ELKS/RAB6-interacting/CAST family member 1 & ERC1 & Mus:NA Hg:The protein encoded by this gene is a member of a family of RIM-binding proteins. RIMs are active zone proteins that regulate neurotransmitter release. This gene has been found fused to the receptor-type tyrosine kinase gene RET by gene rearrangement due to the translocation t(10;12)(q11;p13) in thyroid papillary carcinoma. Alternative splicing results in multiple transcript variants. \\ 
  61 & Fam213a & family with sequence similarity 213, member A & FAM213A & Mus:NA Hg:NA \\ 
  62 & Fry & furry homolog (Drosophila) & FRY & Mus:NA Hg:NA \\ 
  63 & Gas7 & growth arrest specific 7 & GAS7 & Mus:NA Hg:Growth arrest-specific 7 is expressed primarily in terminally differentiated brain cells and predominantly in mature cerebellar Purkinje neurons. GAS7 plays a putative role in neuronal development. Several transcript variants encoding proteins which vary in the N-terminus have been described. \\ 
  65 & Gatm & glycine amidinotransferase (L-arginine:glycine amidinotransferase) & GATM & Mus:NA Hg:This gene encodes a mitochondrial enzyme that belongs to the amidinotransferase family. This enzyme is involved in creatine biosynthesis, whereby it catalyzes the transfer of a guanido group from L-arginine to glycine, resulting in guanidinoacetic acid, the immediate precursor of creatine. Mutations in this gene cause arginine:glycine amidinotransferase deficiency, an inborn error of creatine synthesis characterized by mental retardation, language impairment, and behavioral disorders. \\ 
  66 & Gm10231 & ATP synthase, H+ transporting, mitochondrial F0 complex, subunit C2 (subunit 9) pseudogene &  & Mus:NA Hg:NA \\ 

   \hline
\end{tabular}
\end{table}

\begin{table}[ht]
\centering
\tiny
\begin{tabular}{rlp{3cm}lp{12cm}}
  \hline
 & query & name & hg & summary \\ 
  \hline

  68 & Gm12054 & predicted gene 12054 &  & Mus:NA Hg:NA \\ 
  69 & Gm13394 & predicted gene 13394 &  & Mus:NA Hg:NA \\ 
  71 & Gm15662 & predicted gene 15662 &  & Mus:NA Hg:NA \\ 
  72 & Gm16295 & predicted gene 16295 &  & Mus:NA Hg:NA \\ 
  73 & Gm17131 & predicted gene 17131 &  & Mus:NA Hg:NA \\ 
  74 & Gm20390 & predicted gene 20390 &  & Mus:NA Hg:NA \\ 
  75 & Gm21962 & predicted gene, 21962 &  & Mus:NA Hg:NA \\ 
  76 & Gnat1 & guanine nucleotide binding protein, alpha transducing 1 & GNAT1 & Mus:NA Hg:Transducin is a 3-subunit guanine nucleotide-binding protein (G protein) which stimulates the coupling of rhodopsin and cGMP-phoshodiesterase during visual impulses. The transducin alpha subunits in rods and cones are encoded by separate genes. This gene encodes the alpha subunit in rods. This gene is also expressed in other cells, and has been implicated in bitter taste transduction in rat taste cells. Mutations in this gene result in autosomal dominant congenital stationary night blindness. Multiple alternatively spliced variants, encoding the same protein, have been identified. \\ 
  77 & Gpr17 & G protein-coupled receptor 17 & GPR17 & Mus:NA Hg:NA \\ 
  78 & Gpr26 & G protein-coupled receptor 26 & GPR26 & Mus:NA Hg:NA \\ 
  79 & Grid2ip & glutamate receptor, ionotropic, delta 2 (Grid2) interacting protein 1 & GRID2IP & Mus:NA Hg:Glutamate receptor delta-2 (GRID2; MIM 602368) is predominantly expressed at parallel fiber-Purkinje cell postsynapses and plays crucial roles in synaptogenesis and synaptic plasticity. GRID2IP1 interacts with GRID2 and may control GRID2 signaling in Purkinje cells (Matsuda et al., 2006 [PubMed 16835239]). \\ 
  80 & Grin2a & glutamate receptor, ionotropic, NMDA2A (epsilon 1) & GRIN2A & Mus:NA Hg:This gene encodes a member of the glutamate-gated ion channel protein family. The encoded protein is an N-methyl-D-aspartate (NMDA) receptor subunit. NMDA receptors are both ligand-gated and voltage-dependent, and are involved in long-term potentiation, an activity-dependent increase in the efficiency of synaptic transmission thought to underlie certain kinds of memory and learning. These receptors are permeable to calcium ions, and activation results in a calcium influx into post-synaptic cells, which results in the activation of several signaling cascades. Disruption of this gene is associated with focal epilepsy and speech disorder with or without mental retardation. Alternative splicing results in multiple transcript variants. \\ 
  82 & Gsn & gelsolin & GSN & Mus:NA Hg:The protein encoded by this gene binds to the 'plus' ends of actin monomers and filaments to prevent monomer exchange. The encoded calcium-regulated protein functions in both assembly and disassembly of actin filaments. Defects in this gene are a cause of familial amyloidosis Finnish type (FAF). Multiple transcript variants encoding several different isoforms have been found for this gene. \\ 
  83 & Gtpbp6 & GTP binding protein 6 (putative) & GTPBP6 & Mus:NA Hg:This gene encodes a GTP binding protein and is located in the pseudoautosomal region (PAR) at the end of the short arms of the X and Y chromosomes. \\ 
  84 & H2afj & H2A histone family, member J & H2AFJ & Mus:NA Hg:Histones are basic nuclear proteins that are responsible for the nucleosome structure of the chromosomal fiber in eukaryotes. Nucleosomes consist of approximately 146 bp of DNA wrapped around a histone octamer composed of pairs of each of the four core histones (H2A, H2B, H3, and H4). The chromatin fiber is further compacted through the interaction of a linker histone, H1, with the DNA between the nucleosomes to form higher order chromatin structures. This gene is located on chromosome 12 and encodes a variant H2A histone. The protein is divergent at the C-terminus compared to the consensus H2A histone family member. This gene also encodes an antimicrobial peptide with antibacterial and antifungal activity. \\ 
  85 & Hipk2 & homeodomain interacting protein kinase 2 & HIPK2 & Mus:NA Hg:This gene encodes a conserved serine/threonine kinase that is a member of the homeodomain-interacting protein kinase family. The encoded protein interacts with homeodomain transcription factors and many other transcription factors such as p53, and can function as both a corepressor and a coactivator depending on the transcription factor and its subcellular localization. Multiple transcript variants encoding different isoforms have been found for this gene. \\ 
  86 & Hs6st3 & heparan sulfate 6-O-sulfotransferase 3 & HS6ST3 & Mus:NA Hg:Heparan sulfate (HS) sulfotransferases, such as HS6ST3, modify HS to generate structures required for interactions between HS and a variety of proteins. These interactions are implicated in proliferation and differentiation, adhesion, migration, inflammation, blood coagulation, and other diverse processes (Habuchi et al., 2000 [PubMed 10644753]). \\ 
  87 & Htr2a & 5-hydroxytryptamine (serotonin) receptor 2A & HTR2A & Mus:NA Hg:This gene encodes one of the receptors for serotonin, a neurotransmitter with many roles. Mutations in this gene are associated with susceptibility to schizophrenia and obsessive-compulsive disorder, and are also associated with response to the antidepressant citalopram in patients with major depressive disorder (MDD). MDD patients who also have a mutation in intron 2 of this gene show a significantly reduced response to citalopram as this antidepressant downregulates expression of this gene. Multiple transcript variants encoding different isoforms have been found for this gene. \\ 
  88 & Igsf9b & immunoglobulin superfamily, member 9B & IGSF9B & Mus:NA Hg:NA \\ 

   \hline
\end{tabular}
\end{table}

\begin{table}[ht]
\centering
\tiny
\begin{tabular}{rlp{3cm}lp{12cm}}
  \hline
 & query & name & hg & summary \\ 
  \hline

  89 & Il33 & interleukin 33 & IL33 & Mus:NA Hg:IL33 (MIM 608678) is a member of the IL1 (see MIM 147760) family that potently drives production of T helper-2 (Th2)-associated cytokines (e.g., IL4; MIM 147780). IL33 is a ligand for IL33R (IL1RL1; MIM 601203), an IL1 family receptor that is selectively expressed on Th2 cells and mast cells (summary by Yagami et al., 2010 [PubMed 20926795]). \\ 
  91 & Kcnma1 & potassium large conductance calcium-activated channel, subfamily M, alpha member 1 & KCNMA1 & Mus:NA Hg:MaxiK channels are large conductance, voltage and calcium-sensitive potassium channels which are fundamental to the control of smooth muscle tone and neuronal excitability. MaxiK channels can be formed by 2 subunits: the pore-forming alpha subunit, which is the product of this gene, and the modulatory beta subunit. Intracellular calcium regulates the physical association between the alpha and beta subunits. Alternatively spliced transcript variants encoding different isoforms have been identified. \\ 
  92 & Kcnq3 & potassium voltage-gated channel, subfamily Q, member 3 & KCNQ3 & Mus:NA Hg:This gene encodes a protein that functions in the regulation of neuronal excitability. The encoded protein forms an M-channel by associating with the products of the related KCNQ2 or KCNQ5 genes, which both encode integral membrane proteins. M-channel currents are inhibited by M1 muscarinic acetylcholine receptors and are activated by retigabine, a novel anti-convulsant drug. Defects in this gene are a cause of benign familial neonatal convulsions type 2 (BFNC2), also known as epilepsy, benign neonatal type 2 (EBN2). Alternative splicing of this gene results in multiple transcript variants. \\ 
  93 & Ksr2 & kinase suppressor of ras 2 & KSR2 & Mus:NA Hg:NA \\ 
  94 & Lars2 & leucyl-tRNA synthetase, mitochondrial & LARS2 & Mus:NA Hg:This gene encodes a class 1 aminoacyl-tRNA synthetase, mitochondrial leucyl-tRNA synthetase. Each of the twenty aminoacyl-tRNA synthetases catalyzes the aminoacylation of a specific tRNA or tRNA isoaccepting family with the cognate amino acid. \\ 
  95 & Lsm4 & LSM4 homolog, U6 small nuclear RNA associated (S. cerevisiae) & LSM4 & Mus:NA Hg:This gene encodes a member of the LSm family of RNA-binding proteins. LSm proteins form stable heteromers that bind specifically to the 3'-terminal oligo(U) tract of U6 snRNA and may play a role in pre-mRNA splicing by mediating U4/U6 snRNP formation. Alternatively spliced transcript variants encoding multiple isoforms have been observed for this gene. \\ 
  96 & Ly6h & lymphocyte antigen 6 complex, locus H & LY6H & Mus:NA Hg:NA \\ 
  98 & Madd & MAP-kinase activating death domain & MADD & Mus:NA Hg:Tumor necrosis factor alpha (TNF-alpha) is a signaling molecule that interacts with one of two receptors on cells targeted for apoptosis. The apoptotic signal is transduced inside these cells by cytoplasmic adaptor proteins. The protein encoded by this gene is a death domain-containing adaptor protein that interacts with the death domain of TNF-alpha receptor 1 to activate mitogen-activated protein kinase (MAPK) and propagate the apoptotic signal. It is membrane-bound and expressed at a higher level in neoplastic cells than in normal cells. Several transcript variants encoding different isoforms have been described for this gene. \\ 
  99 & Manf & mesencephalic astrocyte-derived neurotrophic factor & MANF & Mus:NA Hg:The protein encoded by this gene is localized in the endoplasmic reticulum (ER) and golgi, and is also secreted. Reducing expression of this gene increases susceptibility to ER stress-induced death and results in cell proliferation. Activity of this protein is important in promoting the survival of dopaminergic neurons. The presence of polymorphisms in the N-terminal arginine-rich region, including a specific mutation that changes an ATG start codon to AGG, have been reported in a variety of solid tumors; however, these polymorphisms were later shown to exist in normal tissues and are thus no longer thought to be tumor-related. \\ 
  100 & Map4k4 & mitogen-activated protein kinase kinase kinase kinase 4 & MAP4K4 & Mus:NA Hg:The protein encoded by this gene is a member of the serine/threonine protein kinase family. This kinase has been shown to specifically activate MAPK8/JNK. The activation of MAPK8 by this kinase is found to be inhibited by the dominant-negative mutants of MAP3K7/TAK1, MAP2K4/MKK4, and MAP2K7/MKK7, which suggests that this kinase may function through the MAP3K7-MAP2K4-MAP2K7 kinase cascade, and mediate the TNF-alpha signaling pathway. Alternatively spliced transcript variants encoding different isoforms have been identified. \\ 
  101 & Marcksl1 & MARCKS-like 1 & MARCKSL1 & Mus:NA Hg:This gene encodes a member of the myristoylated alanine-rich C-kinase substrate (MARCKS) family. Members of this family play a role in cytoskeletal regulation, protein kinase C signaling and calmodulin signaling. The encoded protein affects the formation of adherens junction. Alternative splicing results in multiple transcript variants. Pseudogenes of this gene are located on the long arm of chromosomes 6 and 10. \\ 
  102 & Mgat5 & mannoside acetylglucosaminyltransferase 5 & MGAT5 & Mus:NA Hg:The protein encoded by this gene belongs to the glycosyltransferase family. It catalyzes the addition of beta-1,6-N-acetylglucosamine to the alpha-linked mannose of biantennary N-linked oligosaccharides present on the newly synthesized glycoproteins. It is one of the most important enzymes involved in the regulation of the biosynthesis of glycoprotein oligosaccharides. Alterations of the oligosaccharides on cell surface glycoproteins cause significant changes in the adhesive or migratory behavior of a cell. Increase in the activity of this enzyme has been correlated with the progression of invasive malignancies. \\ 
  103 & Mical3 & microtubule associated monooxygenase, calponin and LIM domain containing 3 & MICAL3 & Mus:NA Hg:NA \\ 
  104 & Mirg & miRNA containing gene &  & Mus:NA Hg:NA \\ 

   \hline
\end{tabular}
\end{table}

\begin{table}[ht]
\centering
\tiny
\begin{tabular}{rlp{3cm}lp{12cm}}
  \hline
 & query & name & hg & summary \\ 
  \hline

  106 & Mpc2 & mitochondrial pyruvate carrier 2 & MPC2 & Mus:NA Hg:NA \\ 
  107 & Mpeg1 & macrophage expressed gene 1 & MPEG1 & Mus:NA Hg:NA \\ 
  108 & Mpzl1 & myelin protein zero-like 1 & MPZL1 & Mus:NA Hg:NA \\ 
  109 & Mrpl27 & mitochondrial ribosomal protein L27 & MRPL27 & Mus:NA Hg:Mammalian mitochondrial ribosomal proteins are encoded by nuclear genes and help in protein synthesis within the mitochondrion. Mitochondrial ribosomes (mitoribosomes) consist of a small 28S subunit and a large 39S subunit. They have an estimated 75\% protein to rRNA composition compared to prokaryotic ribosomes, where this ratio is reversed. Another difference between mammalian mitoribosomes and prokaryotic ribosomes is that the latter contain a 5S rRNA. Among different species, the proteins comprising the mitoribosome differ greatly in sequence, and sometimes in biochemical properties, which prevents easy recognition by sequence homology. This gene encodes a 39S subunit protein. \\ 
  110 & Msrb1 & methionine sulfoxide reductase B1 & MSRB1 & Mus:This gene encodes a selenoprotein, which contains a selenocysteine (Sec) residue at its active site. The selenocysteine is encoded by the UGA codon that normally signals translation termination. The 3' UTR of selenoprotein genes have a common stem-loop structure, the sec insertion sequence (SECIS), that is necessary for the recognition of UGA as a Sec codon rather than as a stop signal. This protein belongs to the methionine sulfoxide reductase B (MsrB) family, and is localized in the cytosol and nucleus. Hg:This gene encodes a selenoprotein, which contains a selenocysteine (Sec) residue at its active site. The selenocysteine is encoded by the UGA codon that normally signals translation termination. The 3' UTR of selenoprotein genes have a common stem-loop structure, the sec insertion sequence (SECIS), that is necessary for the recognition of UGA as a Sec codon rather than as a stop signal. This protein belongs to the methionine sulfoxide reductase (Msr) protein family which includes repair enzymes that reduce oxidized methionine residues in proteins. The protein encoded by this gene is expressed in a variety of adult and fetal tissues and localizes to the cell nucleus and cytosol. \\ 
  111 & Myh9 & myosin, heavy polypeptide 9, non-muscle & MYH9 & Mus:NA Hg:This gene encodes a conventional non-muscle myosin; this protein should not be confused with the unconventional myosin-9a or 9b (MYO9A or MYO9B). The encoded protein is a myosin IIA heavy chain that contains an IQ domain and a myosin head-like domain which is involved in several important functions, including cytokinesis, cell motility and maintenance of cell shape. Defects in this gene have been associated with non-syndromic sensorineural deafness autosomal dominant type 17, Epstein syndrome, Alport syndrome with macrothrombocytopenia, Sebastian syndrome, Fechtner syndrome and macrothrombocytopenia with progressive sensorineural deafness. \\ 
  112 & Mypop & Myb-related transcription factor, partner of profilin & MYPOP & Mus:NA Hg:NA \\ 
  114 & Ncan & neurocan & NCAN & Mus:NA Hg:Neurocan is a chondroitin sulfate proteoglycan thought to be involved in the modulation of cell adhesion and migration. \\ 
  115 & Ndufa2 & NADH dehydrogenase (ubiquinone) 1 alpha subcomplex, 2 & NDUFA2 & Mus:This gene encodes a subunit of the NADH-ubiquinone oxidoreductase (complex I) enzyme, which is a large, multimeric protein. It is the first enzyme complex in the mitochondrial electron transport chain and catalyzes the transfer of electrons from NADH to the electron acceptor ubiquinone. The proton gradient created by electron transfer drives the conversion of ADP to ATP. The human ortholog of this gene has been characterized, and its structure and redox potential is reported to be similar to that of thioredoxins. It may be involved in regulating complex I activity or assembly via assistance in redox processes. In humans, mutations in this gene are associated with Leigh syndrome, an early-onset progressive neurodegenerative disorder. A pseudogene of this gene is located on chromosome 5. Hg:The encoded protein is a subunit of the hydrophobic protein fraction of the NADH:ubiquinone oxidoreductase (complex 1), the first enzyme complex in the electron transport chain located in the inner mitochondrial membrane, and may be involved in regulating complex I activity or its assembly via assistance in redox processes. Mutations in this gene are associated with Leigh syndrome, an early-onset progressive neurodegenerative disorder. Alternative splicing results in multiple transcript variants. \\ 
  116 & Ndufaf2 & NADH dehydrogenase (ubiquinone) 1 alpha subcomplex, assembly factor 2 & NDUFAF2 & Mus:NA Hg:NADH:ubiquinone oxidoreductase (complex I) catalyzes the transfer of electrons from NADH to ubiquinone (coenzyme Q) in the first step of the mitochondrial respiratory chain, resulting in the translocation of protons across the inner mitochondrial membrane. This gene encodes a complex I assembly factor. Mutations in this gene cause progressive encephalopathy resulting from mitochondrial complex I deficiency. \\ 
  117 & Ndufb2 & NADH dehydrogenase (ubiquinone) 1 beta subcomplex, 2 & NDUFB2 & Mus:NA Hg:The protein encoded by this gene is a subunit of the multisubunit NADH:ubiquinone oxidoreductase (complex I). Mammalian complex I is composed of 45 different subunits. This protein has NADH dehydrogenase activity and oxidoreductase activity. It plays a important role in transfering electrons from NADH to the respiratory chain. The immediate electron acceptor for the enzyme is believed to be ubiquinone. Hydropathy analysis revealed that this subunit and 4 other subunits have an overall hydrophilic pattern, even though they are found within the hydrophobic protein (HP) fraction of complex I. \\ 
  119 & Ndufc2 & NADH dehydrogenase (ubiquinone) 1, subcomplex unknown, 2 & NDUFC2 & Mus:NA Hg:NA \\ 

   \hline
\end{tabular}
\end{table}

\begin{table}[ht]
\centering
\tiny
\begin{tabular}{rlp{3cm}lp{12cm}}
  \hline
 & query & name & hg & summary \\ 
  \hline

  120 & Neat1 & nuclear paraspeckle assembly transcript 1 (non-protein coding) &  & Mus:NA Hg:NA \\ 
  122 & Nenf & neuron derived neurotrophic factor & NENF & Mus:NA Hg:This gene encodes a neurotrophic factor that may play a role in neuron differentiation and development. A pseudogene of this gene is found on chromosome 12. Alternate splicing of this gene results in multiple transcript variants. \\ 
  123 & Nfasc & neurofascin & NFASC & Mus:This gene encodes an L1 family immunoglobulin cell adhesion molecule with multiple IGcam and fibronectin domains. The protein functions in neurite outgrowth, neurite fasciculation, and organization of the axon initial segment (AIS) and nodes of Ranvier on axons during early development. Both the AIS and nodes of Ranvier contain high densities of voltage-gated Na+ (Nav) channels which are clustered by interactions with cytoskeletal and scaffolding proteins including this protein, gliomedin, ankyrin 3 (ankyrin-G), and betaIV spectrin. This protein links the AIS extracellular matrix to the intracellular cytoskeleton. This gene undergoes extensive alternative splicing, and the full-length nature of some variants has not been determined. Hg:This gene encodes an L1 family immunoglobulin cell adhesion molecule with multiple IGcam and fibronectin domains. The protein functions in neurite outgrowth, neurite fasciculation, and organization of the axon initial segment (AIS) and nodes of Ranvier on axons during early development. Both the AIS and nodes of Ranvier contain high densities of voltage-gated Na+ (Nav) channels which are clustered by interactions with cytoskeletal and scaffolding proteins including this protein, gliomedin, ankyrin 3 (ankyrin-G), and betaIV spectrin. This protein links the AIS extracellular matrix to the intracellular cytoskeleton. This gene undergoes extensive alternative splicing, and the full-length nature of some variants has not been determined. \\ 
  124 & Nkx6-2 & NK6 homeobox 2 & NKX6-2 & Mus:NA Hg:NA \\ 
  125 & Npnt & nephronectin & NPNT & Mus:NA Hg:NA \\ 
  126 & Nrip1 & nuclear receptor interacting protein 1 & NRIP1 & Mus:NA Hg:Nuclear receptor interacting protein 1 (NRIP1) is a nuclear protein that specifically interacts with the hormone-dependent activation domain AF2 of nuclear receptors. Also known as RIP140, this protein modulates transcriptional activity of the estrogen receptor. \\ 
  128 & Ntrk3 & neurotrophic tyrosine kinase, receptor, type 3 & NTRK3 & Mus:NA Hg:This gene encodes a member of the neurotrophic tyrosine receptor kinase (NTRK) family. This kinase is a membrane-bound receptor that, upon neurotrophin binding, phosphorylates itself and members of the MAPK pathway. Signalling through this kinase leads to cell differentiation and may play a role in the development of proprioceptive neurons that sense body position. Mutations in this gene have been associated with medulloblastomas, secretory breast carcinomas and other cancers. Several transcript variants encoding different isoforms have been found for this gene. \\ 
  129 & Oaz1 & ornithine decarboxylase antizyme 1 & OAZ1 & Mus:The protein encoded by this gene belongs to the ornithine decarboxylase antizyme family, which plays a role in cell growth and proliferation by regulating intracellular polyamine levels. Expression of antizymes requires +1 ribosomal frameshifting, which is enhanced by high levels of polyamines. Antizymes in turn bind to and inhibit ornithine decarboxylase (ODC), the key enzyme in polyamine biosynthesis; thus, completing the auto-regulatory circuit. This gene encodes antizyme 1, the first member of the antizyme family, that has broad tissue distribution, and negatively regulates intracellular polyamine levels by binding to and targeting ODC for degradation, as well as inhibiting polyamine uptake. Antizyme 1 mRNA contains two potential in-frame AUGs; and studies in rat suggest that alternative use of the two translation initiation sites results in N-terminally distinct protein isoforms with different subcellular localization. Alternatively spliced transcript variants have also been noted for this gene. Hg:The protein encoded by this gene belongs to the ornithine decarboxylase antizyme family, which plays a role in cell growth and proliferation by regulating intracellular polyamine levels. Expression of antizymes requires +1 ribosomal frameshifting, which is enhanced by high levels of polyamines. Antizymes in turn bind to and inhibit ornithine decarboxylase (ODC), the key enzyme in polyamine biosynthesis; thus, completing the auto-regulatory circuit. This gene encodes antizyme 1, the first member of the antizyme family, that has broad tissue distribution, and negatively regulates intracellular polyamine levels by binding to and targeting ODC for degradation, as well as inhibiting polyamine uptake. Antizyme 1 mRNA contains two potential in-frame AUGs; and studies in rat suggest that alternative use of the two translation initiation sites results in N-terminally distinct protein isoforms with different subcellular localization. Alternatively spliced transcript variants have also been noted for this gene. \\ 
  130 & P2ry12 & purinergic receptor P2Y, G-protein coupled 12 & P2RY12 & Mus:NA Hg:The product of this gene belongs to the family of G-protein coupled receptors. This family has several receptor subtypes with different pharmacological selectivity, which overlaps in some cases, for various adenosine and uridine nucleotides. This receptor is involved in platelet aggregation, and is a potential target for the treatment of thromboembolisms and other clotting disorders. Mutations in this gene are implicated in bleeding disorder, platelet type 8 (BDPLT8). Alternative splicing results in multiple transcript variants of this gene. \\ 
  132 & Pcdhb9 & protocadherin beta 9 &  & Mus:NA Hg:NA \\ 

   \hline
\end{tabular}
\end{table}

\begin{table}[ht]
\centering
\tiny
\begin{tabular}{rlp{3cm}lp{12cm}}
  \hline
 & query & name & hg & summary \\ 
  \hline

  133 & Pcsk1n & proprotein convertase subtilisin/kexin type 1 inhibitor & PCSK1N & Mus:NA Hg:The protein encoded by this gene functions as an inhibitor of prohormone convertase 1, which regulates the proteolytic cleavage of neuroendocrine peptide precursors. The proprotein is further processed into multiple short peptides. A polymorphism within this gene may be associated with obesity. \\ 
  134 & Piga & phosphatidylinositol glycan anchor biosynthesis, class A & PIGA & Mus:NA Hg:This gene encodes a protein required for synthesis of N-acetylglucosaminyl phosphatidylinositol (GlcNAc-PI), the first intermediate in the biosynthetic pathway of GPI anchor. The GPI anchor is a glycolipid found on many blood cells and which serves to anchor proteins to the cell surface. Paroxysmal nocturnal hemoglobinuria, an acquired hematologic disorder, has been shown to result from mutations in this gene. Alternate splice variants have been characterized. A related pseudogene is located on chromosome 12. \\ 
  136 & Plxna2 & plexin A2 & PLXNA2 & Mus:NA Hg:This gene encodes a member of the plexin-A family of semaphorin co-receptors. Semaphorins are a large family of secreted or membrane-bound proteins that mediate repulsive effects on axon pathfinding during nervous system development. A subset of semaphorins are recognized by plexin-A/neuropilin transmembrane receptor complexes, triggering a cellular signal transduction cascade that leads to axon repulsion. This plexin-A family member is thought to transduce signals from semaphorin-3A and -3C. \\ 
  137 & Plxna3 & plexin A3 & PLXNA3 & Mus:NA Hg:This gene encodes a member of the plexin class of proteins. The encoded protein is a class 3 semaphorin receptor, and may be involved in cytoskeletal remodeling and as well as apoptosis. Studies of a similar gene in zebrafish suggest that it is important for axon pathfinding in the developing nervous system. This gene may be associated with tumor progression. \\ 
  138 & Plxna4 & plexin A4 & PLXNA4 & Mus:NA Hg:NA \\ 
  139 & Pmm1 & phosphomannomutase 1 & PMM1 & Mus:NA Hg:Phosphomannomutase catalyzes the conversion between D-mannose 6-phosphate and D-mannose 1-phosphate which is a substrate for GDP-mannose synthesis. GDP-mannose is used for synthesis of dolichol-phosphate-mannose, which is essential for N-linked glycosylation and thus the secretion of several glycoproteins as well as for the synthesis of glycosyl-phosphatidyl-inositol (GPI) anchored proteins. \\ 
  140 & Ppp1r12a & protein phosphatase 1, regulatory (inhibitor) subunit 12A & PPP1R12A & Mus:NA Hg:Myosin phosphatase target subunit 1, which is also called the myosin-binding subunit of myosin phosphatase, is one of the subunits of myosin phosphatase. Myosin phosphatase regulates the interaction of actin and myosin downstream of the guanosine triphosphatase Rho. The small guanosine triphosphatase Rho is implicated in myosin light chain (MLC) phosphorylation, which results in contraction of smooth muscle and interaction of actin and myosin in nonmuscle cells. The guanosine triphosphate (GTP)-bound, active form of RhoA (GTP.RhoA) specifically interacted with the myosin-binding subunit (MBS) of myosin phosphatase, which regulates the extent of phosphorylation of MLC. Rho-associated kinase (Rho-kinase), which is activated by GTP. RhoA, phosphorylated MBS and consequently inactivated myosin phosphatase. Overexpression of RhoA or activated RhoA in NIH 3T3 cells increased phosphorylation of MBS and MLC. Thus, Rho appears to inhibit myosin phosphatase through the action of Rho-kinase. Several transcript variants encoding different isoforms have been found for this gene. \\ 
  141 & Ppp1r35 & protein phosphatase 1, regulatory subunit 35 & PPP1R35 & Mus:NA Hg:NA \\ 
  142 & Prr24 &  & PRR24 & Mus:NA Hg:NA \\ 
  143 & Ptprj & protein tyrosine phosphatase, receptor type, J & PTPRJ & Mus:NA Hg:The protein encoded by this gene is a member of the protein tyrosine phosphatase (PTP) family. PTPs are known to be signaling molecules that regulate a variety of cellular processes, including cell growth, differentiation, mitotic cycle, and oncogenic transformation. This PTP possesses an extracellular region containing five fibronectin type III repeats, a single transmembrane region, and a single intracytoplasmic catalytic domain, and thus represents a receptor-type PTP. This protein is present in all hematopoietic lineages, and was shown to negatively regulate T cell receptor signaling possibly through interfering with the phosphorylation of Phospholipase C Gamma 1 and Linker for Activation of T Cells. This protein can also dephosphorylate the PDGF beta receptor, and may be involved in UV-induced signal transduction. Multiple transcript variants encoding different isoforms have been found for this gene. \\ 
  144 & Rabac1 & Rab acceptor 1 (prenylated) & RABAC1 & Mus:NA Hg:NA \\ 
  145 & Rho & rhodopsin & RHO & Mus:NA Hg:Retinitis pigmentosa is an inherited progressive disease which is a major cause of blindness in western communities. It can be inherited as an autosomal dominant, autosomal recessive, or X-linked recessive disorder. In the autosomal dominant form,which comprises about 25\% of total cases, approximately 30\% of families have mutations in the gene encoding the rod photoreceptor-specific protein rhodopsin. This is the transmembrane protein which, when photoexcited, initiates the visual transduction cascade. Defects in this gene are also one of the causes of congenital stationary night blindness. \\ 
  146 & Rims4 & regulating synaptic membrane exocytosis 4 & RIMS4 & Mus:NA Hg:NA \\ 

   \hline
\end{tabular}
\end{table}

\begin{table}[ht]
\centering
\tiny
\begin{tabular}{rlp{3cm}lp{12cm}}
  \hline
 & query & name & hg & summary \\ 
  \hline

  147 & Rora & RAR-related orphan receptor alpha & RORA & Mus:The protein encoded by this gene is a member of the NR1 subfamily of nuclear hormone receptors. It can bind as a monomer or as a homodimer to hormone response elements upstream of several genes to enhance the expression of those genes. The encoded protein has been shown to interact with NM23-2, a nucleoside diphosphate kinase involved in organogenesis and differentiation, as well as with NM23-1, the product of a tumor metastasis suppressor candidate gene. Also, it has been shown to aid in the transcriptional regulation of some genes involved in circadian rhythm. Three transcript variants encoding different isoforms have been found for this gene. Hg:The protein encoded by this gene is a member of the NR1 subfamily of nuclear hormone receptors. It can bind as a monomer or as a homodimer to hormone response elements upstream of several genes to enhance the expression of those genes. The encoded protein has been shown to interact with NM23-2, a nucleoside diphosphate kinase involved in organogenesis and differentiation, as well as with NM23-1, the product of a tumor metastasis suppressor candidate gene. Also, it has been shown to aid in the transcriptional regulation of some genes involved in circadian rhythm. Four transcript variants encoding different isoforms have been described for this gene. \\ 
  148 & RP23-81C12.3 &  &  & Mus:NA Hg:NA \\ 
  149 & Rpl13 & ribosomal protein L13 & RPL13 & Mus:NA Hg:Ribosomes, the organelles that catalyze protein synthesis, consist of a small 40S subunit and a large 60S subunit. Together these subunits are composed of 4 RNA species and approximately 80 structurally distinct proteins. This gene encodes a ribosomal protein that is a component of the 60S subunit. The protein belongs to the L13E family of ribosomal proteins. It is located in the cytoplasm. This gene is expressed at significantly higher levels in benign breast lesions than in breast carcinomas. Alternatively spliced transcript variants encoding distinct isoforms have been found for this gene. As is typical for genes encoding ribosomal proteins, there are multiple processed pseudogenes of this gene dispersed through the genome. \\ 
  150 & Rpl18a & ribosomal protein L18A & RPL18A & Mus:NA Hg:Ribosomes, the organelles that catalyze protein synthesis, consist of a small 40S subunit and a large 60S subunit. Together these subunits are composed of 4 RNA species and approximately 80 structurally distinct proteins. This gene encodes a member of the L18AE family of ribosomal proteins that is a component of the 60S subunit. The encoded protein may play a role in viral replication by interacting with the hepatitis C virus internal ribosome entry site (IRES). This gene is co-transcribed with the U68 snoRNA, located within the third intron. As is typical for genes encoding ribosomal proteins, there are multiple processed pseudogenes of this gene dispersed throughout the genome. \\ 
  151 & Rpl22l1 & ribosomal protein L22 like 1 &  & Mus:NA Hg:NA \\ 
  152 & Rpl30 & ribosomal protein L30 & RPL30 & Mus:NA Hg:Ribosomes, the organelles that catalyze protein synthesis, consist of a small 40S subunit and a large 60S subunit. Together these subunits are composed of 4 RNA species and approximately 80 structurally distinct proteins. This gene encodes a ribosomal protein that is a component of the 60S subunit. The protein belongs to the L30E family of ribosomal proteins. It is located in the cytoplasm. This gene is co-transcribed with the U72 small nucleolar RNA gene, which is located in its fourth intron. As is typical for genes encoding ribosomal proteins, there are multiple processed pseudogenes of this gene dispersed through the genome. \\ 
  153 & Rpl35 & ribosomal protein L35 & RPL35 & Mus:NA Hg:Ribosomes, the organelles that catalyze protein synthesis, consist of a small 40S subunit and a large 60S subunit. Together these subunits are composed of 4 RNA species and approximately 80 structurally distinct proteins. This gene encodes a ribosomal protein that is a component of the 60S subunit. The protein belongs to the L29P family of ribosomal proteins. It is located in the cytoplasm. As is typical for genes encoding ribosomal proteins, there are multiple processed pseudogenes of this gene dispersed through the genome. \\ 
  154 & Rpl37 & ribosomal protein L37 & RPL37 & Mus:NA Hg:Ribosomes, the organelles that catalyze protein synthesis, consist of a small 40S subunit and a large 60S subunit. Together these subunits are composed of 4 RNA species and approximately 80 structurally distinct proteins. This gene encodes a ribosomal protein that is a component of the 60S subunit. The protein belongs to the L37E family of ribosomal proteins. It is located in the cytoplasm. The protein contains a C2C2-type zinc finger-like motif. As is typical for genes encoding ribosomal proteins, there are multiple processed pseudogenes of this gene dispersed through the genome. \\ 
  155 & Rpl39 & ribosomal protein L39 & RPL39 & Mus:NA Hg:Ribosomes, the organelles that catalyze protein synthesis, consist of a small 40S subunit and a large 60S subunit. Together these subunits are composed of 4 RNA species and approximately 80 structurally distinct proteins. This gene encodes a ribosomal protein that is a component of the 60S subunit. The protein belongs to the S39E family of ribosomal proteins. It is located in the cytoplasm. In rat, the protein is the smallest, and one of the most basic, proteins of the ribosome. This gene is co-transcribed with the U69 small nucleolar RNA gene, which is located in its second intron. As is typical for genes encoding ribosomal proteins, there are multiple processed pseudogenes of this gene dispersed through the genome. \\ 

   \hline
\end{tabular}
\end{table}

\begin{table}[ht]
\centering
\tiny
\begin{tabular}{rlp{3cm}lp{12cm}}
  \hline
 & query & name & hg & summary \\ 
  \hline

  156 & Rplp2 & ribosomal protein, large P2 & RPLP2 & Mus:NA Hg:Ribosomes, the organelles that catalyze protein synthesis, consist of a small 40S subunit and a large 60S subunit. Together these subunits are composed of 4 RNA species and approximately 80 structurally distinct proteins. This gene encodes a ribosomal phosphoprotein that is a component of the 60S subunit. The protein, which is a functional equivalent of the E. coli L7/L12 ribosomal protein, belongs to the L12P family of ribosomal proteins. It plays an important role in the elongation step of protein synthesis. Unlike most ribosomal proteins, which are basic, the encoded protein is acidic. Its C-terminal end is nearly identical to the C-terminal ends of the ribosomal phosphoproteins P0 and P1. The P2 protein can interact with P0 and P1 to form a pentameric complex consisting of P1 and P2 dimers, and a P0 monomer. The protein is located in the cytoplasm. As is typical for genes encoding ribosomal proteins, there are multiple processed pseudogenes of this gene dispersed through the genome. \\ 
  157 & Rprd2 & regulation of nuclear pre-mRNA domain containing 2 & RPRD2 & Mus:NA Hg:NA \\ 
  158 & Rps19 & ribosomal protein S19 & RPS19 & Mus:NA Hg:Ribosomes, the organelles that catalyze protein synthesis, consist of a small 40S subunit and a large 60S subunit. Together these subunits are composed of 4 RNA species and approximately 80 structurally distinct proteins. This gene encodes a ribosomal protein that is a component of the 40S subunit. The protein belongs to the S19E family of ribosomal proteins. It is located in the cytoplasm. Mutations in this gene cause Diamond-Blackfan anemia (DBA), a constitutional erythroblastopenia characterized by absent or decreased erythroid precursors, in a subset of patients. This suggests a possible extra-ribosomal function for this gene in erythropoietic differentiation and proliferation, in addition to its ribosomal function. Higher expression levels of this gene in some primary colon carcinomas compared to matched normal colon tissues has been observed. As is typical for genes encoding ribosomal proteins, there are multiple processed pseudogenes of this gene dispersed through the genome. \\ 
  160 & Rps21 & ribosomal protein S21 & RPS21 & Mus:NA Hg:Ribosomes, the organelles that catalyze protein synthesis, consist of a small 40S subunit and a large 60S subunit. Together these subunits are composed of 4 RNA species and approximately 80 structurally distinct proteins. This gene encodes a ribosomal protein that is a component of the 40S subunit. The protein belongs to the S21E family of ribosomal proteins. It is located in the cytoplasm. Alternative splice variants that encode different protein isoforms have been described, but their existence has not been verified. As is typical for genes encoding ribosomal proteins, there are multiple processed pseudogenes of this gene dispersed through the genome. \\ 
  161 & Rps5 & ribosomal protein S5 & RPS5 & Mus:NA Hg:Ribosomes, the organelles that catalyze protein synthesis, consist of a small 40S subunit and a large 60S subunit. Together these subunits are composed of 4 RNA species and approximately 80 structurally distinct proteins. This gene encodes a ribosomal protein that is a component of the 40S subunit. The protein belongs to the S7P family of ribosomal proteins. It is located in the cytoplasm. Variable expression of this gene in colorectal cancers compared to adjacent normal tissues has been observed, although no correlation between the level of expression and the severity of the disease has been found. As is typical for genes encoding ribosomal proteins, there are multiple processed pseudogenes of this gene dispersed through the genome. \\ 
  163 & Sbno1 & sno, strawberry notch homolog 1 (Drosophila) & SBNO1 & Mus:NA Hg:NA \\ 
  164 & Scand1 & SCAN domain-containing 1 & SCAND1 & Mus:NA Hg:This gene encodes a SCAN box domain-containing protein. The SCAN domain is a highly conserved, leucine-rich motif of approximately 60 aa originally found within a subfamily of zinc finger proteins. This gene belongs to a family of genes that encode an isolated SCAN domain, but no zinc finger motif. This protein binds to and may regulate the function of the transcription factor myeloid zinc finger 1B. Alternate splicing results in multiple transcript variants. \\ 
  165 & Sema4d & sema domain, immunoglobulin domain (Ig), transmembrane domain (TM) and short cytoplasmic domain, (semaphorin) 4D & SEMA4D & Mus:NA Hg:NA \\ 
  167 & Sept11 & septin 11 & SEPT11 & Mus:NA Hg:SEPT11 belongs to the conserved septin family of filament-forming cytoskeletal GTPases that are involved in a variety of cellular functions including cytokinesis and vesicle trafficking (Hanai et al., 2004 [PubMed 15196925]; Nagata et al., 2004 [PubMed 15485874]). \\ 
  168 & Shc3 & src homology 2 domain-containing transforming protein C3 & SHC3 & Mus:NA Hg:NA \\ 
  169 & Slc5a5 & solute carrier family 5 (sodium iodide symporter), member 5 & SLC5A5 & Mus:NA Hg:This gene encodes a member of the sodium glucose cotransporter family. The encoded protein is responsible for the uptake of iodine in tissues such as the thyroid and lactating breast tissue. The iodine taken up by the thyroid is incorporated into the metabolic regulators triiodothyronine (T3) and tetraiodothyronine (T4). Mutations in this gene are associated with thyroid dyshormonogenesis 1. \\ 

   \hline
\end{tabular}
\end{table}

\begin{table}[ht]
\centering
\tiny
\begin{tabular}{rlp{3cm}lp{12cm}}
  \hline
 & query & name & hg & summary \\ 
  \hline

  170 & Slc8a1 & solute carrier family 8 (sodium/calcium exchanger), member 1 & SLC8A1 & Mus:NA Hg:In cardiac myocytes, Ca(2+) concentrations alternate between high levels during contraction and low levels during relaxation. The increase in Ca(2+) concentration during contraction is primarily due to release of Ca(2+) from intracellular stores. However, some Ca(2+) also enters the cell through the sarcolemma (plasma membrane). During relaxation, Ca(2+) is sequestered within the intracellular stores. To prevent overloading of intracellular stores, the Ca(2+) that entered across the sarcolemma must be extruded from the cell. The Na(+)-Ca(2+) exchanger is the primary mechanism by which the Ca(2+) is extruded from the cell during relaxation. In the heart, the exchanger may play a key role in digitalis action. The exchanger is the dominant mechanism in returning the cardiac myocyte to its resting state following excitation. \\ 
  171 & Smdt1 & single-pass membrane protein with aspartate rich tail 1 & SMDT1 & Mus:NA Hg:NA \\ 
  172 & Smpd3 & sphingomyelin phosphodiesterase 3, neutral & SMPD3 & Mus:NA Hg:NA \\ 
  173 & Socs7 & suppressor of cytokine signaling 7 & SOCS7 & Mus:NA Hg:NA \\ 
  174 & Spint2 & serine protease inhibitor, Kunitz type 2 & SPINT2 & Mus:NA Hg:This gene encodes a transmembrane protein with two extracellular Kunitz domains that inhibits a variety of serine proteases. The protein inhibits HGF activator which prevents the formation of active hepatocyte growth factor. This gene is a putative tumor suppressor, and mutations in this gene result in congenital sodium diarrhea. Multiple transcript variants encoding different isoforms have been found for this gene. \\ 
  175 & Stard4 & StAR-related lipid transfer (START) domain containing 4 & STARD4 & Mus:NA Hg:Cholesterol homeostasis is regulated, at least in part, by sterol regulatory element (SRE)-binding proteins (e.g., SREBP1; MIM 184756) and by liver X receptors (e.g., LXRA; MIM 602423). Upon sterol depletion, LXRs are inactive and SREBPs are cleaved, after which they bind promoter SREs and activate genes involved in cholesterol biosynthesis and uptake. Sterol transport is mediated by vesicles or by soluble protein carriers, such as steroidogenic acute regulatory protein (STAR; MIM 600617). STAR is homologous to a family of proteins containing a 200- to 210-amino acid STAR-related lipid transfer (START) domain, including STARD4 (Soccio et al., 2002 [PubMed 12011452]). \\ 
  176 & Strn & striatin, calmodulin binding protein & STRN & Mus:NA Hg:NA \\ 
  177 & Syvn1 & synovial apoptosis inhibitor 1, synoviolin & SYVN1 & Mus:NA Hg:This gene encodes a protein involved in endoplasmic reticulum (ER)-associated degradation. The encoded protein removes unfolded proteins, accumulated during ER stress, by retrograde transport to the cytosol from the ER. This protein also uses the ubiquitin-proteasome system for additional degradation of unfolded proteins. Sequence analysis identified two transcript variants that encode different isoforms. \\ 
  179 & Tef & thyrotroph embryonic factor & TEF & Mus:NA Hg:This gene encodes a member of the PAR (proline and acidic amino acid-rich) subfamily of basic region/leucine zipper (bZIP) transcription factors. It is expressed in a broad range of cells and tissues in adult animals, however, during embryonic development, TEF expression appears to be restricted to the developing anterior pituitary gland, coincident with the appearance of thyroid-stimulating hormone, beta (TSHB). Indeed, TEF can bind to, and transactivate the TSHB promoter. It shows homology (in the functional domains) with other members of the PAR-bZIP subfamily of transcription factors, which include albumin D box-binding protein (DBP), human hepatic leukemia factor (HLF) and chicken vitellogenin gene-binding protein (VBP); VBP is considered the chicken homologue of TEF. Different members of the subfamily can readily form heterodimers, and share DNA-binding, and transcriptional regulatory properties. Alternatively spliced transcript variants encoding different isoforms have been found for this gene. \\ 
  180 & Tet3 & tet methylcytosine dioxygenase 3 & TET3 & Mus:NA Hg:Members of the ten-eleven translocation (TET) gene family, including TET3, play a role in the DNA methylation process (Langemeijer et al., 2009 [PubMed 19923888]). \\ 
  181 & Tmem179 & transmembrane protein 179 & TMEM179 & Mus:NA Hg:NA \\ 
  182 & Tmem242 & transmembrane protein 242 & TMEM242 & Mus:NA Hg:NA \\ 
  183 & Tnfrsf22 & tumor necrosis factor receptor superfamily, member 22 &  & Mus:NA Hg:NA \\ 
  184 & Tnnc1 & troponin C, cardiac/slow skeletal & TNNC1 & Mus:NA Hg:Troponin is a central regulatory protein of striated muscle contraction, and together with tropomyosin, is located on the actin filament. Troponin consists of 3 subunits: TnI, which is the inhibitor of actomyosin ATPase; TnT, which contains the binding site for tropomyosin; and TnC, the protein encoded by this gene. The binding of calcium to TnC abolishes the inhibitory action of TnI, thus allowing the interaction of actin with myosin, the hydrolysis of ATP, and the generation of tension. Mutations in this gene are associated with cardiomyopathy dilated type 1Z. \\ 

   \hline
\end{tabular}
\end{table}

\begin{table}[ht]
\centering
\tiny
\begin{tabular}{rlp{3cm}lp{12cm}}
  \hline
 & query & name & hg & summary \\ 
  \hline

  185 & Tomm6 & translocase of outer mitochondrial membrane 6 homolog (yeast) & TOMM6 & Mus:NA Hg:NA \\ 
  186 & Tprkb & Tp53rk binding protein & TPRKB & Mus:NA Hg:NA \\ 
  188 & Trem2 & triggering receptor expressed on myeloid cells 2 & TREM2 & Mus:The protein encoded by this gene is part of the immunoglobulin and lectin-like superfamily and functions as part of the innate immune system. This gene forms part of a cluster of genes on mouse chromosome 17 thought to be involved in innate immunity. This protein associates with the adaptor protein Dap-12 and recruits several factors, such as kinases and phospholipase C-gamma, to form a receptor signaling complex that activates myeloid cells, including dendritic cells and microglia. In humans homozygous loss-of-function mutations in this gene cause Nasu-Hakola disease and mutations in this gene may be risk factors to the development of Alzheimer's disease. In mouse mutations of this gene serve as a pathophysiological model for polycystic lipomembranous osteodysplasia with sclerosing leukoencephalopathy (Nasu-Hakola disease) and for inflammatory bowel disease. Alternative splicing results in multiple transcript variants that encode different protein isoforms. Hg:This gene encodes a membrane protein that forms a receptor signaling complex with the TYRO protein tyrosine kinase binding protein. The encoded protein functions in immune response and may be involved in chronic inflammation by triggering the production of constitutive inflammatory cytokines. Defects in this gene are a cause of polycystic lipomembranous osteodysplasia with sclerosing leukoencephalopathy (PLOSL). Alternative splicing results in multiple transcript variants encoding different isoforms. \\ 
  189 & Tspan2 & tetraspanin 2 & TSPAN2 & Mus:NA Hg:The protein encoded by this gene is a member of the transmembrane 4 superfamily, also known as the tetraspanin family. Most of these members are cell-surface proteins that are characterized by the presence of four hydrophobic domains. The proteins mediate signal transduction events that play a role in the regulation of cell development, activation, growth and motility. \\ 
  190 & Txlna & taxilin alpha & TXLNA & Mus:NA Hg:NA \\ 
  192 & Tyrobp & TYRO protein tyrosine kinase binding protein & TYROBP & Mus:NA Hg:This gene encodes a transmembrane signaling polypeptide which contains an immunoreceptor tyrosine-based activation motif (ITAM) in its cytoplasmic domain. The encoded protein may associate with the killer-cell inhibitory receptor (KIR) family of membrane glycoproteins and may act as an activating signal transduction element. This protein may bind zeta-chain (TCR) associated protein kinase 70kDa (ZAP-70) and spleen tyrosine kinase (SYK) and play a role in signal transduction, bone modeling, brain myelination, and inflammation. Mutations within this gene have been associated with polycystic lipomembranous osteodysplasia with sclerosing leukoencephalopathy (PLOSL), also known as Nasu-Hakola disease. Its putative receptor, triggering receptor expressed on myeloid cells 2 (TREM2), also causes PLOSL. Multiple alternative transcript variants encoding distinct isoforms have been identified for this gene. \\ 
  193 & Uba52 & ubiquitin A-52 residue ribosomal protein fusion product 1 &  & Mus:NA Hg:NA \\ 
  194 & Vmn1r58 & vomeronasal 1 receptor 58 &  & Mus:NA Hg:NA \\ 
  195 & Vps37d & vacuolar protein sorting 37D (yeast) & VPS37D & Mus:NA Hg:NA \\ 
  196 & Vstm2l & V-set and transmembrane domain containing 2-like & VSTM2L & Mus:NA Hg:NA \\ 
  197 & Zdhhc9 & zinc finger, DHHC domain containing 9 & ZDHHC9 & Mus:NA Hg:This gene encodes an integral membrane protein that is a member of the zinc finger DHHC domain-containing protein family. The encoded protein forms a complex with golgin subfamily A member 7 and functions as a palmitoyltransferase. This protein specifically palmitoylates HRAS and NRAS. Mutations in this gene are associated with X-linked mental retardation. Alternate splicing results in multiple transcript variants that encode the same protein. \\ 
  198 & Zfp941 & zinc finger protein 941 &  & Mus:NA Hg:NA \\ 
   \hline
\end{tabular}
\end{table}

\end{document}