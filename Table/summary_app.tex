\documentclass[11pt, landscape]{article}   	% use "amsart" instead of "article" for AMSLaTeX format
\usepackage{geometry}                		% See geometry.pdf to learn the layout options. There are lots.
\geometry{letterpaper}                   		% ... or a4paper or a5paper or ... 
%\geometry{landscape}                		% Activate for for rotated page geometry
%\usepackage[parfill]{parskip}    		% Activate to begin paragraphs with an empty line rather than an indent
\usepackage{graphicx}				% Use pdf, png, jpg, or eps§ with pdflatex; use eps in DVI mode
								% TeX will automatically convert eps --> pdf in pdflatex		
\usepackage{amssymb}

\title{Brief information of the genes}
\author{AD project}
%\date{}							% Activate to display a given date or no date

\begin{document}
\maketitle
%\section{}
%\subsection{}

% latex table generated in R 3.1.1 by xtable 1.7-4 package
% Fri Jan 23 17:00:39 2015
\begin{table}[ht]
\centering
\tiny
\begin{tabular}{rlp{3cm}lp{12cm}}
  \hline
 & query & name & hg & summary \\ 
  \hline
1 & 1700047I17Rik2 & RIKEN cDNA 1700047I17 gene 2 &  & Mus:NA Hg:NA \\ 
  2 & 2310036O22Rik & RIKEN cDNA 2310036O22 gene & C19orf43 & Mus:NA Hg:NA \\ 
  3 & 2900079G21Rik & RIKEN cDNA 2900079G21 gene &  & Mus:NA Hg:NA \\ 
  5 & App & amyloid beta (A4) precursor protein & APP & Mus:NA Hg:This gene encodes a cell surface receptor and transmembrane precursor protein that is cleaved by secretases to form a number of peptides. Some of these peptides are secreted and can bind to the acetyltransferase complex APBB1/TIP60 to promote transcriptional activation, while others form the protein basis of the amyloid plaques found in the brains of patients with Alzheimer disease. In addition, two of the peptides are antimicrobial peptides, having been shown to have bacteriocidal and antifungal activities. Mutations in this gene have been implicated in autosomal dominant Alzheimer disease and cerebroarterial amyloidosis (cerebral amyloid angiopathy). Multiple transcript variants encoding several different isoforms have been found for this gene. \\ 
  6 & Arpp21 & cyclic AMP-regulated phosphoprotein, 21 & ARPP21 & Mus:NA Hg:This gene encodes a cAMP-regulated phosphoprotein. The encoded protein is enriched in the caudate nucleus and cerebellar cortex. A similar protein in mouse may be involved in regulating the effects of dopamine in the basal ganglia. Alternate splicing results in multiple transcript variants. \\ 
  7 & Asl & argininosuccinate lyase & ASL & Mus:NA Hg:This gene encodes a member of the lyase 1 family. The encoded protein forms a cytosolic homotetramer and primarily catalyzes the reversible hydrolytic cleavage of argininosuccinate into arginine and fumarate, an essential step in the liver in detoxifying ammonia via the urea cycle. Mutations in this gene result in the autosomal recessive disorder argininosuccinic aciduria, or argininosuccinic acid lyase deficiency. A nontranscribed pseudogene is also located on the long arm of chromosome 22. Alternatively spliced transcript variants encoding different isoforms have been described. \\ 
  9 & Cacna1e & calcium channel, voltage-dependent, R type, alpha 1E subunit & CACNA1E & Mus:This gene encodes an integral membrane protein that belongs to the calcium channel alpha-1 subunits family. Voltage-sensitive calcium channels mediate the entry of calcium ions into excitable cells and are also involved in a variety of calcium-dependent processes. Voltage-dependent calcium channels are multi-subunit complexes, comprised of alpha-1, alpha-2, beta and delta subunits in a 1:1:1:1 ratio. The isoform alpha-1E gives rise to R-type calcium currents and belongs to the high-voltage activated group. Calcium channels containing the alpha-1E subunit may be involved in the modulation of neuronal firing patterns, an important component of information processing. Hg:Voltage-dependent calcium channels are multisubunit complexes consisting of alpha-1, alpha-2, beta, and delta subunits in a 1:1:1:1 ratio. These channels mediate the entry of calcium ions into excitable cells, and are also involved in a variety of calcium-dependent processes, including muscle contraction, hormone or neurotransmitter release, gene expression, cell motility, cell division and cell death. This gene encodes the alpha-1E subunit of the R-type calcium channels, which belong to the 'high-voltage activated' group that maybe involved in the modulation of firing patterns of neurons important for information processing. Alternatively spliced transcript variants encoding different isoforms have been described for this gene. \\ 
  10 & Ccdc124 & coiled-coil domain containing 124 & CCDC124 & Mus:NA Hg:NA \\ 
  11 & Ccdc85b & coiled-coil domain containing 85B & CCDC85B & Mus:NA Hg:Hepatitis delta virus (HDV) is a pathogenic human virus whose RNA genome and replication cycle resemble those of plant viroids. Delta-interacting protein A (DIPA), a cellular gene product, has been found to have homology to hepatitis delta virus antigen (HDAg). DIPA interacts with the viral antigen, HDAg, and can affect HDV replication in vitro. \\ 
  12 & Col4a1 & collagen, type IV, alpha 1 & COL4A1 & Mus:NA Hg:This gene encodes a type IV collagen alpha protein. Type IV collagen proteins are integral components of basement membranes. This gene shares a bidirectional promoter with a paralogous gene on the opposite strand. The protein consists of an amino-terminal 7S domain, a triple-helix forming collagenous domain, and a carboxy-terminal non-collagenous domain. It functions as part of a heterotrimer and interacts with other extracellular matrix components such as perlecans, proteoglycans, and laminins. In addition, proteolytic cleavage of the non-collagenous carboxy-terminal domain results in a biologically active fragment known as arresten, which has anti-angiogenic and tumor suppressor properties. Mutations in this gene cause porencephaly, cerebrovascular disease, and renal and muscular defects. Alternative splicing results in multiple transcript variants. \\ 
  13 & Cox17 & cytochrome c oxidase assembly protein 17 & COX17 & Mus:NA Hg:Cytochrome c oxidase (COX), the terminal component of the mitochondrial respiratory chain, catalyzes the electron transfer from reduced cytochrome c to oxygen. This component is a heteromeric complex consisting of 3 catalytic subunits encoded by mitochondrial genes and multiple structural subunits encoded by nuclear genes. The mitochondrially-encoded subunits function in electron transfer, and the nuclear-encoded subunits may function in the regulation and assembly of the complex. This nuclear gene encodes a protein which is not a structural subunit, but may be involved in the recruitment of copper to mitochondria for incorporation into the COX apoenzyme. This protein shares 92\% amino acid sequence identity with mouse and rat Cox17 proteins. This gene is no longer considered to be a candidate gene for COX deficiency. A pseudogene COX17P has been found on chromosome 13. \\ 

   \hline
\end{tabular}
\end{table}

\begin{table}[ht]
\centering
\tiny
\begin{tabular}{rlp{3cm}lp{12cm}}
  \hline
 & query & name & hg & summary \\ 
  \hline

  14 & Crebbp & CREB binding protein & CREBBP & Mus:NA Hg:This gene is ubiquitously expressed and is involved in the transcriptional coactivation of many different transcription factors. First isolated as a nuclear protein that binds to cAMP-response element binding protein (CREB), this gene is now known to play critical roles in embryonic development, growth control, and homeostasis by coupling chromatin remodeling to transcription factor recognition. The protein encoded by this gene has intrinsic histone acetyltransferase activity and also acts as a scaffold to stabilize additional protein interactions with the transcription complex. This protein acetylates both histone and non-histone proteins. This protein shares regions of very high sequence similarity with protein p300 in its bromodomain, cysteine-histidine-rich regions, and histone acetyltransferase domain. Mutations in this gene cause Rubinstein-Taybi syndrome (RTS). Chromosomal translocations involving this gene have been associated with acute myeloid leukemia. Alternative splicing results in multiple transcript variants encoding different isoforms. \\ 
  15 & Dio2 & deiodinase, iodothyronine, type II & DIO2 & Mus:The protein encoded by this gene belongs to the iodothyronine deiodinase family. It activates thyroid hormone by converting the prohormone thyroxine (T4) by outer ring deiodination (ORD) to bioactive 3,3',5-triiodothyronine (T3). Knockout studies in mice suggest that this gene may play a role in brown adipose tissue lipogenesis, auditory function, and bone formation. This protein contains selenocysteine (Sec) residues encoded by the UGA codon, which normally signals translation termination. The 3' UTR of Sec-containing genes have a common stem-loop structure, the sec insertion sequence (SECIS), which is necessary for the recognition of UGA as a Sec codon rather than as a stop signal. Hg:The protein encoded by this gene belongs to the iodothyronine deiodinase family. It activates thyroid hormone by converting the prohormone thyroxine (T4) by outer ring deiodination (ORD) to bioactive 3,3',5-triiodothyronine (T3). It is highly expressed in the thyroid, and may contribute significantly to the relative increase in thyroidal T3 production in patients with Graves disease and thyroid adenomas. This protein contains selenocysteine (Sec) residues encoded by the UGA codon, which normally signals translation termination. The 3' UTR of Sec-containing genes have a common stem-loop structure, the sec insertion sequence (SECIS), which is necessary for the recognition of UGA as a Sec codon rather than as a stop signal. Alternative splicing results in multiple transcript variants encoding different isoforms. \\ 
  16 & Dnajb11 & DnaJ (Hsp40) homolog, subfamily B, member 11 & DNAJB11 & Mus:NA Hg:This gene encodes a soluble glycoprotein of the endoplasmic reticulum (ER) lumen that functions as a co-chaperone of binding immunoglobulin protein, a 70 kilodalton heat shock protein chaperone required for the proper folding and assembly of proteins in the ER. The encoded protein contains a highly conserved J domain of about 70 amino acids with a characteristic His-Pro-Asp (HPD) motif and may regulate the activity of binding immunoglobulin protein by stimulating ATPase activity. \\ 
  18 & Ech1 & enoyl coenzyme A hydratase 1, peroxisomal & ECH1 & Mus:NA Hg:This gene encodes a member of the hydratase/isomerase superfamily. The gene product shows high sequence similarity to enoyl-coenzyme A (CoA) hydratases of several species, particularly within a conserved domain characteristic of these proteins. The encoded protein, which contains a C-terminal peroxisomal targeting sequence, localizes to the peroxisome. The rat ortholog, which localizes to the matrix of both the peroxisome and mitochondria, can isomerize 3-trans,5-cis-dienoyl-CoA to 2-trans,4-trans-dienoyl-CoA, indicating that it is a delta3,5-delta2,4-dienoyl-CoA isomerase. This enzyme functions in the auxiliary step of the fatty acid beta-oxidation pathway. Expression of the rat gene is induced by peroxisome proliferators. \\ 
  19 & Gm17131 & predicted gene 17131 &  & Mus:NA Hg:NA \\ 
  20 & Gm20390 & predicted gene 20390 &  & Mus:NA Hg:NA \\ 
  21 & Gm21962 & predicted gene, 21962 &  & Mus:NA Hg:NA \\ 
  22 & Gnat1 & guanine nucleotide binding protein, alpha transducing 1 & GNAT1 & Mus:NA Hg:Transducin is a 3-subunit guanine nucleotide-binding protein (G protein) which stimulates the coupling of rhodopsin and cGMP-phoshodiesterase during visual impulses. The transducin alpha subunits in rods and cones are encoded by separate genes. This gene encodes the alpha subunit in rods. This gene is also expressed in other cells, and has been implicated in bitter taste transduction in rat taste cells. Mutations in this gene result in autosomal dominant congenital stationary night blindness. Multiple alternatively spliced variants, encoding the same protein, have been identified. \\ 
  23 & Grid2ip & glutamate receptor, ionotropic, delta 2 (Grid2) interacting protein 1 & GRID2IP & Mus:NA Hg:Glutamate receptor delta-2 (GRID2; MIM 602368) is predominantly expressed at parallel fiber-Purkinje cell postsynapses and plays crucial roles in synaptogenesis and synaptic plasticity. GRID2IP1 interacts with GRID2 and may control GRID2 signaling in Purkinje cells (Matsuda et al., 2006 [PubMed 16835239]). \\ 
  24 & Grin2a & glutamate receptor, ionotropic, NMDA2A (epsilon 1) & GRIN2A & Mus:NA Hg:This gene encodes a member of the glutamate-gated ion channel protein family. The encoded protein is an N-methyl-D-aspartate (NMDA) receptor subunit. NMDA receptors are both ligand-gated and voltage-dependent, and are involved in long-term potentiation, an activity-dependent increase in the efficiency of synaptic transmission thought to underlie certain kinds of memory and learning. These receptors are permeable to calcium ions, and activation results in a calcium influx into post-synaptic cells, which results in the activation of several signaling cascades. Disruption of this gene is associated with focal epilepsy and speech disorder with or without mental retardation. Alternative splicing results in multiple transcript variants. \\ 
  26 & Gtpbp6 & GTP binding protein 6 (putative) & GTPBP6 & Mus:NA Hg:This gene encodes a GTP binding protein and is located in the pseudoautosomal region (PAR) at the end of the short arms of the X and Y chromosomes. \\ 

   \hline
\end{tabular}
\end{table}

\begin{table}[ht]
\centering
\tiny
\begin{tabular}{rlp{3cm}lp{12cm}}
  \hline
 & query & name & hg & summary \\ 
  \hline

  27 & H2afj & H2A histone family, member J & H2AFJ & Mus:NA Hg:Histones are basic nuclear proteins that are responsible for the nucleosome structure of the chromosomal fiber in eukaryotes. Nucleosomes consist of approximately 146 bp of DNA wrapped around a histone octamer composed of pairs of each of the four core histones (H2A, H2B, H3, and H4). The chromatin fiber is further compacted through the interaction of a linker histone, H1, with the DNA between the nucleosomes to form higher order chromatin structures. This gene is located on chromosome 12 and encodes a variant H2A histone. The protein is divergent at the C-terminus compared to the consensus H2A histone family member. This gene also encodes an antimicrobial peptide with antibacterial and antifungal activity. \\ 
  28 & Htr2a & 5-hydroxytryptamine (serotonin) receptor 2A & HTR2A & Mus:NA Hg:This gene encodes one of the receptors for serotonin, a neurotransmitter with many roles. Mutations in this gene are associated with susceptibility to schizophrenia and obsessive-compulsive disorder, and are also associated with response to the antidepressant citalopram in patients with major depressive disorder (MDD). MDD patients who also have a mutation in intron 2 of this gene show a significantly reduced response to citalopram as this antidepressant downregulates expression of this gene. Multiple transcript variants encoding different isoforms have been found for this gene. \\ 
  29 & Lamr1-ps1 & laminin receptor 1 (ribosomal protein SA), pseudogene 1 &  & Mus:NA Hg:NA \\ 
  31 & Lsm4 & LSM4 homolog, U6 small nuclear RNA associated (S. cerevisiae) & LSM4 & Mus:NA Hg:This gene encodes a member of the LSm family of RNA-binding proteins. LSm proteins form stable heteromers that bind specifically to the 3'-terminal oligo(U) tract of U6 snRNA and may play a role in pre-mRNA splicing by mediating U4/U6 snRNP formation. Alternatively spliced transcript variants encoding multiple isoforms have been observed for this gene. \\ 
  32 & Manf & mesencephalic astrocyte-derived neurotrophic factor & MANF & Mus:NA Hg:The protein encoded by this gene is localized in the endoplasmic reticulum (ER) and golgi, and is also secreted. Reducing expression of this gene increases susceptibility to ER stress-induced death and results in cell proliferation. Activity of this protein is important in promoting the survival of dopaminergic neurons. The presence of polymorphisms in the N-terminal arginine-rich region, including a specific mutation that changes an ATG start codon to AGG, have been reported in a variety of solid tumors; however, these polymorphisms were later shown to exist in normal tissues and are thus no longer thought to be tumor-related. \\ 
  33 & Mical3 & microtubule associated monooxygenase, calponin and LIM domain containing 3 & MICAL3 & Mus:NA Hg:NA \\ 
  34 & Mirg & miRNA containing gene &  & Mus:NA Hg:NA \\ 
  36 & Mpeg1 & macrophage expressed gene 1 & MPEG1 & Mus:NA Hg:NA \\ 
  37 & Mpzl1 & myelin protein zero-like 1 & MPZL1 & Mus:NA Hg:NA \\ 
  38 & Mrpl27 & mitochondrial ribosomal protein L27 & MRPL27 & Mus:NA Hg:Mammalian mitochondrial ribosomal proteins are encoded by nuclear genes and help in protein synthesis within the mitochondrion. Mitochondrial ribosomes (mitoribosomes) consist of a small 28S subunit and a large 39S subunit. They have an estimated 75\% protein to rRNA composition compared to prokaryotic ribosomes, where this ratio is reversed. Another difference between mammalian mitoribosomes and prokaryotic ribosomes is that the latter contain a 5S rRNA. Among different species, the proteins comprising the mitoribosome differ greatly in sequence, and sometimes in biochemical properties, which prevents easy recognition by sequence homology. This gene encodes a 39S subunit protein. \\ 
  39 & Myh9 & myosin, heavy polypeptide 9, non-muscle & MYH9 & Mus:NA Hg:This gene encodes a conventional non-muscle myosin; this protein should not be confused with the unconventional myosin-9a or 9b (MYO9A or MYO9B). The encoded protein is a myosin IIA heavy chain that contains an IQ domain and a myosin head-like domain which is involved in several important functions, including cytokinesis, cell motility and maintenance of cell shape. Defects in this gene have been associated with non-syndromic sensorineural deafness autosomal dominant type 17, Epstein syndrome, Alport syndrome with macrothrombocytopenia, Sebastian syndrome, Fechtner syndrome and macrothrombocytopenia with progressive sensorineural deafness. \\ 
  40 & Ndufa2 & NADH dehydrogenase (ubiquinone) 1 alpha subcomplex, 2 & NDUFA2 & Mus:This gene encodes a subunit of the NADH-ubiquinone oxidoreductase (complex I) enzyme, which is a large, multimeric protein. It is the first enzyme complex in the mitochondrial electron transport chain and catalyzes the transfer of electrons from NADH to the electron acceptor ubiquinone. The proton gradient created by electron transfer drives the conversion of ADP to ATP. The human ortholog of this gene has been characterized, and its structure and redox potential is reported to be similar to that of thioredoxins. It may be involved in regulating complex I activity or assembly via assistance in redox processes. In humans, mutations in this gene are associated with Leigh syndrome, an early-onset progressive neurodegenerative disorder. A pseudogene of this gene is located on chromosome 5. Hg:The encoded protein is a subunit of the hydrophobic protein fraction of the NADH:ubiquinone oxidoreductase (complex 1), the first enzyme complex in the electron transport chain located in the inner mitochondrial membrane, and may be involved in regulating complex I activity or its assembly via assistance in redox processes. Mutations in this gene are associated with Leigh syndrome, an early-onset progressive neurodegenerative disorder. Alternative splicing results in multiple transcript variants. \\ 

   \hline
\end{tabular}
\end{table}

\begin{table}[ht]
\centering
\tiny
\begin{tabular}{rlp{3cm}lp{12cm}}
  \hline
 & query & name & hg & summary \\ 
  \hline

  41 & Ndufaf2 & NADH dehydrogenase (ubiquinone) 1 alpha subcomplex, assembly factor 2 & NDUFAF2 & Mus:NA Hg:NADH:ubiquinone oxidoreductase (complex I) catalyzes the transfer of electrons from NADH to ubiquinone (coenzyme Q) in the first step of the mitochondrial respiratory chain, resulting in the translocation of protons across the inner mitochondrial membrane. This gene encodes a complex I assembly factor. Mutations in this gene cause progressive encephalopathy resulting from mitochondrial complex I deficiency. \\ 
  42 & Ndufb2 & NADH dehydrogenase (ubiquinone) 1 beta subcomplex, 2 & NDUFB2 & Mus:NA Hg:The protein encoded by this gene is a subunit of the multisubunit NADH:ubiquinone oxidoreductase (complex I). Mammalian complex I is composed of 45 different subunits. This protein has NADH dehydrogenase activity and oxidoreductase activity. It plays a important role in transfering electrons from NADH to the respiratory chain. The immediate electron acceptor for the enzyme is believed to be ubiquinone. Hydropathy analysis revealed that this subunit and 4 other subunits have an overall hydrophilic pattern, even though they are found within the hydrophobic protein (HP) fraction of complex I. \\ 
  44 & Ndufc2 & NADH dehydrogenase (ubiquinone) 1, subcomplex unknown, 2 & NDUFC2 & Mus:NA Hg:NA \\ 
  45 & Nsdhl & NAD(P) dependent steroid dehydrogenase-like & NSDHL & Mus:NA Hg:The protein encoded by this gene is localized in the endoplasmic reticulum and is involved in cholesterol biosynthesis. Mutations in this gene are associated with CHILD syndrome, which is a X-linked dominant disorder of lipid metabolism with disturbed cholesterol biosynthesis, and typically lethal in males. Alternatively spliced transcript variants with differing 5' UTR have been found for this gene. \\ 
  46 & Ntrk3 & neurotrophic tyrosine kinase, receptor, type 3 & NTRK3 & Mus:NA Hg:This gene encodes a member of the neurotrophic tyrosine receptor kinase (NTRK) family. This kinase is a membrane-bound receptor that, upon neurotrophin binding, phosphorylates itself and members of the MAPK pathway. Signalling through this kinase leads to cell differentiation and may play a role in the development of proprioceptive neurons that sense body position. Mutations in this gene have been associated with medulloblastomas, secretory breast carcinomas and other cancers. Several transcript variants encoding different isoforms have been found for this gene. \\ 
  47 & Piga & phosphatidylinositol glycan anchor biosynthesis, class A & PIGA & Mus:NA Hg:This gene encodes a protein required for synthesis of N-acetylglucosaminyl phosphatidylinositol (GlcNAc-PI), the first intermediate in the biosynthetic pathway of GPI anchor. The GPI anchor is a glycolipid found on many blood cells and which serves to anchor proteins to the cell surface. Paroxysmal nocturnal hemoglobinuria, an acquired hematologic disorder, has been shown to result from mutations in this gene. Alternate splice variants have been characterized. A related pseudogene is located on chromosome 12. \\ 
  49 & Plxna3 & plexin A3 & PLXNA3 & Mus:NA Hg:This gene encodes a member of the plexin class of proteins. The encoded protein is a class 3 semaphorin receptor, and may be involved in cytoskeletal remodeling and as well as apoptosis. Studies of a similar gene in zebrafish suggest that it is important for axon pathfinding in the developing nervous system. This gene may be associated with tumor progression. \\ 
  50 & Plxna4 & plexin A4 & PLXNA4 & Mus:NA Hg:NA \\ 
  51 & Pmm1 & phosphomannomutase 1 & PMM1 & Mus:NA Hg:Phosphomannomutase catalyzes the conversion between D-mannose 6-phosphate and D-mannose 1-phosphate which is a substrate for GDP-mannose synthesis. GDP-mannose is used for synthesis of dolichol-phosphate-mannose, which is essential for N-linked glycosylation and thus the secretion of several glycoproteins as well as for the synthesis of glycosyl-phosphatidyl-inositol (GPI) anchored proteins. \\ 
  52 & Prnp & prion protein & PRNP & Mus:NA Hg:The protein encoded by this gene is a membrane glycosylphosphatidylinositol-anchored glycoprotein that tends to aggregate into rod-like structures. The encoded protein contains a highly unstable region of five tandem octapeptide repeats. This gene is found on chromosome 20, approximately 20 kbp upstream of a gene which encodes a biochemically and structurally similar protein to the one encoded by this gene. Mutations in the repeat region as well as elsewhere in this gene have been associated with Creutzfeldt-Jakob disease, fatal familial insomnia, Gerstmann-Straussler disease, Huntington disease-like 1, and kuru. An overlapping open reading frame has been found for this gene that encodes a smaller, structurally unrelated protein, AltPrp. Alternative splicing results in multiple transcript variants. \\ 
  53 & Psen1 & presenilin 1 & PSEN1 & Mus:NA Hg:Alzheimer's disease (AD) patients with an inherited form of the disease carry mutations in the presenilin proteins (PSEN1; PSEN2) or in the amyloid precursor protein (APP). These disease-linked mutations result in increased production of the longer form of amyloid-beta (main component of amyloid deposits found in AD brains). Presenilins are postulated to regulate APP processing through their effects on gamma-secretase, an enzyme that cleaves APP. Also, it is thought that the presenilins are involved in the cleavage of the Notch receptor, such that they either directly regulate gamma-secretase activity or themselves are protease enzymes. Several alternatively spliced transcript variants encoding different isoforms have been identified for this gene, the full-length nature of only some have been determined. \\ 
  54 & Rho & rhodopsin & RHO & Mus:NA Hg:Retinitis pigmentosa is an inherited progressive disease which is a major cause of blindness in western communities. It can be inherited as an autosomal dominant, autosomal recessive, or X-linked recessive disorder. In the autosomal dominant form,which comprises about 25\% of total cases, approximately 30\% of families have mutations in the gene encoding the rod photoreceptor-specific protein rhodopsin. This is the transmembrane protein which, when photoexcited, initiates the visual transduction cascade. Defects in this gene are also one of the causes of congenital stationary night blindness. \\ 
   \hline
\end{tabular}
\end{table}

\begin{table}[ht]
\centering
\tiny
\begin{tabular}{rlp{3cm}lp{12cm}}
  \hline
 & query & name & hg & summary \\ 
  \hline

  55 & Rpl35 & ribosomal protein L35 & RPL35 & Mus:NA Hg:Ribosomes, the organelles that catalyze protein synthesis, consist of a small 40S subunit and a large 60S subunit. Together these subunits are composed of 4 RNA species and approximately 80 structurally distinct proteins. This gene encodes a ribosomal protein that is a component of the 60S subunit. The protein belongs to the L29P family of ribosomal proteins. It is located in the cytoplasm. As is typical for genes encoding ribosomal proteins, there are multiple processed pseudogenes of this gene dispersed through the genome. \\ 
  56 & Rps19 & ribosomal protein S19 & RPS19 & Mus:NA Hg:Ribosomes, the organelles that catalyze protein synthesis, consist of a small 40S subunit and a large 60S subunit. Together these subunits are composed of 4 RNA species and approximately 80 structurally distinct proteins. This gene encodes a ribosomal protein that is a component of the 40S subunit. The protein belongs to the S19E family of ribosomal proteins. It is located in the cytoplasm. Mutations in this gene cause Diamond-Blackfan anemia (DBA), a constitutional erythroblastopenia characterized by absent or decreased erythroid precursors, in a subset of patients. This suggests a possible extra-ribosomal function for this gene in erythropoietic differentiation and proliferation, in addition to its ribosomal function. Higher expression levels of this gene in some primary colon carcinomas compared to matched normal colon tissues has been observed. As is typical for genes encoding ribosomal proteins, there are multiple processed pseudogenes of this gene dispersed through the genome. \\ 
  58 & Rps21 & ribosomal protein S21 & RPS21 & Mus:NA Hg:Ribosomes, the organelles that catalyze protein synthesis, consist of a small 40S subunit and a large 60S subunit. Together these subunits are composed of 4 RNA species and approximately 80 structurally distinct proteins. This gene encodes a ribosomal protein that is a component of the 40S subunit. The protein belongs to the S21E family of ribosomal proteins. It is located in the cytoplasm. Alternative splice variants that encode different protein isoforms have been described, but their existence has not been verified. As is typical for genes encoding ribosomal proteins, there are multiple processed pseudogenes of this gene dispersed through the genome. \\ 
  59 & Slc8a1 & solute carrier family 8 (sodium/calcium exchanger), member 1 & SLC8A1 & Mus:NA Hg:In cardiac myocytes, Ca(2+) concentrations alternate between high levels during contraction and low levels during relaxation. The increase in Ca(2+) concentration during contraction is primarily due to release of Ca(2+) from intracellular stores. However, some Ca(2+) also enters the cell through the sarcolemma (plasma membrane). During relaxation, Ca(2+) is sequestered within the intracellular stores. To prevent overloading of intracellular stores, the Ca(2+) that entered across the sarcolemma must be extruded from the cell. The Na(+)-Ca(2+) exchanger is the primary mechanism by which the Ca(2+) is extruded from the cell during relaxation. In the heart, the exchanger may play a key role in digitalis action. The exchanger is the dominant mechanism in returning the cardiac myocyte to its resting state following excitation. \\ 
  60 & Smdt1 & single-pass membrane protein with aspartate rich tail 1 & SMDT1 & Mus:NA Hg:NA \\ 
  61 & Spint2 & serine protease inhibitor, Kunitz type 2 & SPINT2 & Mus:NA Hg:This gene encodes a transmembrane protein with two extracellular Kunitz domains that inhibits a variety of serine proteases. The protein inhibits HGF activator which prevents the formation of active hepatocyte growth factor. This gene is a putative tumor suppressor, and mutations in this gene result in congenital sodium diarrhea. Multiple transcript variants encoding different isoforms have been found for this gene. \\ 
  62 & Stard4 & StAR-related lipid transfer (START) domain containing 4 & STARD4 & Mus:NA Hg:Cholesterol homeostasis is regulated, at least in part, by sterol regulatory element (SRE)-binding proteins (e.g., SREBP1; MIM 184756) and by liver X receptors (e.g., LXRA; MIM 602423). Upon sterol depletion, LXRs are inactive and SREBPs are cleaved, after which they bind promoter SREs and activate genes involved in cholesterol biosynthesis and uptake. Sterol transport is mediated by vesicles or by soluble protein carriers, such as steroidogenic acute regulatory protein (STAR; MIM 600617). STAR is homologous to a family of proteins containing a 200- to 210-amino acid STAR-related lipid transfer (START) domain, including STARD4 (Soccio et al., 2002 [PubMed 12011452]). \\ 
  63 & Syvn1 & synovial apoptosis inhibitor 1, synoviolin & SYVN1 & Mus:NA Hg:This gene encodes a protein involved in endoplasmic reticulum (ER)-associated degradation. The encoded protein removes unfolded proteins, accumulated during ER stress, by retrograde transport to the cytosol from the ER. This protein also uses the ubiquitin-proteasome system for additional degradation of unfolded proteins. Sequence analysis identified two transcript variants that encode different isoforms. \\ 
  65 & Tef & thyrotroph embryonic factor & TEF & Mus:NA Hg:This gene encodes a member of the PAR (proline and acidic amino acid-rich) subfamily of basic region/leucine zipper (bZIP) transcription factors. It is expressed in a broad range of cells and tissues in adult animals, however, during embryonic development, TEF expression appears to be restricted to the developing anterior pituitary gland, coincident with the appearance of thyroid-stimulating hormone, beta (TSHB). Indeed, TEF can bind to, and transactivate the TSHB promoter. It shows homology (in the functional domains) with other members of the PAR-bZIP subfamily of transcription factors, which include albumin D box-binding protein (DBP), human hepatic leukemia factor (HLF) and chicken vitellogenin gene-binding protein (VBP); VBP is considered the chicken homologue of TEF. Different members of the subfamily can readily form heterodimers, and share DNA-binding, and transcriptional regulatory properties. Alternatively spliced transcript variants encoding different isoforms have been found for this gene. \\ 
   \hline
\end{tabular}
\end{table}

\begin{table}[ht]
\centering
\tiny
\begin{tabular}{rlp{3cm}lp{12cm}}
  \hline
 & query & name & hg & summary \\ 
  \hline

  66 & Tmem242 & transmembrane protein 242 & TMEM242 & Mus:NA Hg:NA \\ 
  67 & Tnfrsf22 & tumor necrosis factor receptor superfamily, member 22 &  & Mus:NA Hg:NA \\ 
  68 & Tomm6 & translocase of outer mitochondrial membrane 6 homolog (yeast) & TOMM6 & Mus:NA Hg:NA \\ 
  69 & Uba52 & ubiquitin A-52 residue ribosomal protein fusion product 1 &  & Mus:NA Hg:NA \\ 
  70 & Vps37d & vacuolar protein sorting 37D (yeast) & VPS37D & Mus:NA Hg:NA \\ 
  71 & Zfp941 & zinc finger protein 941 &  & Mus:NA Hg:NA \\ 
   \hline
\end{tabular}
\end{table}

\end{document}  
